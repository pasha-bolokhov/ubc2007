\documentclass[pdf,UofT06talk,slideColor,colorBG,accumulate]{prosper}
%\usepackage[dvips]{color}
\usepackage{latexsym,pstricks,pst-node,pst-coil,epsf}
\usepackage{amsmath,amsfonts,amsbsy,amssymb}

%geoffs
%blends
%bruce
%autumn
%contemporain
%lignesbleues
%gyom

% Definition of new colors
\newrgbcolor{LemonChiffon}{1. 0.98 0.8}
\newrgbcolor{LightBlue}{0.68 0.85 0.9}
\newrgbcolor{PaleGray}{0.8 0.8 0.8}
\newrgbcolor{PaleRed}{0.9 0.6 0.6}
\newrgbcolor{PaleGreen}{0.4 0.7 0.4}
\newrgbcolor{PaleBlue}{0.4 0.4 0.7}


%\def\fntb{\fontfamily{ppl}\fontshape{it}\fontsize{8pt}{10pt}\selectfont}
%\def\fnt{\usefont{T1}{pcr}{b}{n}\fontsize{8pt}{10pt}\selectfont}

%\def\fnt{\fontfamily{ppl}\selectfont}

%\def\aaa{\fontfamily{ppl}\fontsize{12pt}{6pt}\selectfont}

%\def\fnt{\fontfamily{ppl}\fontsize{8pt}{6pt}\selectfont}

\newcommand{\nc}{\newcommand}
\newcommand{\slashed}[1]{\hbox{{$#1$}\llap{$/$}}}
\newcommand{\p}{\partial}
\newcommand{\mc}[1]{\mathcal{#1}}
\newcommand{\wt}{\widetilde}
\newcommand{\ov}{\overline}
\newcommand{\md}{\mathcal{D}}
\newcommand{\eq}{{\rm eq}}
\newcommand{\lgr}{\left\lgroup}
\newcommand{\rgr}{\right\rgroup}
\newcommand{\GeV}{{\rm GeV}}
\newcommand{\Mpl}{M_{\rm Pl}}
\newcommand{\Gsph}{\Gamma_{\rm sph}}
\newcommand{\meff}{m^{\rm eff}_\nu}
\newcommand{\eV}{{\rm eV}}


\nc{\fr}[2]{\frac{#1}{#2}}
\newrgbcolor{mathcolor}{0.7 0.0 0.7}

\newcommand{\addSlide}[3]
{
 \untilSlide*{#1}{\PaleGray{#3}}
 \fromSlide*{#2}{#3}
}

\slideCaption{Lorentz Violation and Generation of Baryon Asymmetry
		of the Universe}

\title{
\vspace{-1.0cm}
%\vspace{-4.0cm}
%\hspace{-2.0cm}
%\hspace{-20.0cm}
%\begin{flushleft}
\begin{center}
%\includegraphics[height=7cm,keepaspectratio,clip=true]
%		{SanDiego/san4.eps}
\end{center}
%\end{flushleft}
%\vspace{-6.2cm}
\Large\lbrown Lorentz Violation \\ 
			and \\
	Generation of Baryon Asymmetry \\
		of the Universe\\
\normalsize hep-ph/0610070
}
\institution{
%\vspace{-3.0cm}
%\includegraphics[width=12cm]{SanDiego/SDNight-a.eps}
%\it Perimeter Institute for Theoretical Physics
%\includegraphics[width=10cm]{pi.eps}
%\includegraphics[width=10cm]{IMGP1870.EPS}
%\includegraphics[width=9cm]{script.eps}
\vspace{0.7cm}
\brown
University of British Columbia	\quad $ \bullet $ \quad   2007
}

\author{
\vspace{1.4cm}
\brown Pavel A. Bolokhov\\
University of Victoria~~ and ~~St.Petersburg State University
}
 
\begin{document}
\maketitle

%%%%%%%%%%%%%%%%%%%%%%%%%%%%%%  SLIDE %%%%%%%%%%%%%%%%%%%%%%%%%%%%%%%%%%%%%%%
\begin{slide}{}

	\vspace{1.0cm}
\usefont{T1}{ppl}{m}{n}\fontsize{30.0pt}{30.0pt}\selectfont
\begin{center}
	Part I\\
	LV Interactions in the Standard Model
\end{center}

\end{slide}


%%%%%%%%%%%%%%%%%%%%%%%%%%%%%%  SLIDE %%%%%%%%%%%%%%%%%%%%%%%%%%%%%%%%%%%%%%%
\begin{slide}[Glitter]{Introduction to LV}

	Study of Lorentz-violating theories provides probes of physics
	{\mybf far} beyond the Planck scale.
	
	The most straightforward idea to test Lorentz invariance is the
	experiments of Michelson-Morley type. 
	They have pushed New Physics beyond $ 10^{27}~\GeV $ scale. 

	Later came the idea of introducing {\myit kinematic} Lorentz violation,
	i.e. modification of propagation of particles:
{
\mathcolor
\[
	E^2 ~=~ m^2 ~+~ p^2 ~+~ \frac{\eta}{M} \cdot p^3 ~+~ 
			\frac{\kappa} {M^2} \cdot p^4 ~+~ \dots ~.
\]
}
	
	Here astrophysical observations have pushed  $ M $ to 
	$ 10^{33}~\GeV $ level.

	The theories of this sort introduce LV in an {\myit ad hoc} manner.

\end{slide}


%%%%%%%%%%%%%%%%%%%%%%%%%%%%%%  SLIDE %%%%%%%%%%%%%%%%%%%%%%%%%%%%%%%%%%%%%%%
\begin{slide}{}

	In the {\mybf Effective Field Theory} one introduces LV operators

{
\mathcolor
\begin{equation*}
	\mc{L}_{\rm LV} ~=~ \mc{L}_{\rm SM} 
				~+~
			\sum_n 
			c_n^{\mu\nu ...}
			\frac {\mc{O}_{\mu\nu ...}^n} {\Lambda^n}~.
\end{equation*}
}

	This facilitates the study of observable effects caused by explicit
	violation of Lorentz symmetry.

	The coefficients $ c_n^{\mu\nu ...} $ are background tensors which
	break explicitly Lorentz invariance.

	Such a theory may be viewed as a natural effective description of a 
	{\myit spontaneously} broken Lorentz invariance.

	The latter assumption allows to alleviate most of the conceptual questions
	about Lorentz violation.

\end{slide}

%%%%%%%%%%%%%%%%%%%%%%%%%%%%%%  SLIDE %%%%%%%%%%%%%%%%%%%%%%%%%%%%%%%%%%%%%%%
\begin{slide}{Common Sense Requirements}

	The operators should better be 
\begin{itemize}
	\item Gauge invariant
	\item Lorentz invariant, apart from coupling to the background tensor
	\item not reducible to lower dimension operators by the equations
		of motion
	\item not reducible to a total derivative
	\item coupled to an irreducible background tensor.
\end{itemize}

\end{slide}


%%%%%%%%%%%%%%%%%%%%%%%%%%%%%%  SLIDE %%%%%%%%%%%%%%%%%%%%%%%%%%%%%%%%%%%%%%%
%\overlays{4}
%{
\begin{slide}{Lower-dimensional LV interactions}

%\FromSlide{1}
	In QED, the generic expansion in terms of the gauge invariant 
	operators starts at dimension {\red three}:

{
\mathcolor
\[
\nonumber
{\cal L}_{\rm QED}^{(3)} =
~-~{\dgreen a_\mu}\,  \bar \Psi \gamma_\mu \Psi
~-~ {\dgreen b_\mu}\,  \bar \Psi \gamma^\mu \gamma_5 \Psi 
~-~ \frac{1}{2}{\dgreen H_{\mu\nu}}\bar \Psi \sigma^{\mu\nu} \Psi
~-~ {\dgreen k_\mu}\,  
\epsilon^{\mu\nu\kappa\lambda} A_\nu \partial_\kappa A_\lambda~.
%\label{LVqed}
\]
}

%\FromSlide{2}
	Dimension three operators create a problem:
	from dimensional counting one expects 
	{\blue $ a_\mu \sim M n_\mu $}, 
	where {\blue $ n_\mu $} is a unit vector, and
	$ M $ is the scale of New Physics.
	That creates {\myit\blue disastrous} effects.

%\FromSlide{3}
	Even if the LV coefficients are tuned small, they will be 
	exploded by quantum corrections coming from quadratic divergencies
	of the higher-dimensional operators:
%\FromSlide{4}
{
\mathcolor
\[
[LV]_{\rm dim~3} ~~\sim~~ ({\rm loop~factor}) \, 
\;\Lambda_{UV}^2\;
\times\; [LV]_{{\rm dim}~5}~. 
\]
}

\end{slide}
%}

%%%%%%%%%%%%%%%%%%%%%%%%%%%%%%  SLIDE %%%%%%%%%%%%%%%%%%%%%%%%%%%%%%%%%%%%%%%
\begin{slide}{Higher-dimensional LV interactions}

	Higher-dimensional operators start with dimension {\mybf five}.

	Simple operators have been considered before, for instance
\[
	\mc{L}^{\rm dim~5}_{\rm QED} ~~=~~ 	
		C^{\mu\nu\rho}\cdot F_{\mu\lambda} \p_\nu \wt{F}_\rho^{~\lambda}
		~~+~~
		C_1^{\mu\nu\rho}\cdot \ov{\psi}\, \gamma_{(\mu} 
		\mathcal{D}_\nu \mathcal{D}_{\rho)} \psi 
\]

	These particular operators modify dispersion relations of the photon and electron.
	
	Full classification of dimension five interactions has not been done.

	In QED alone one finds 15 operators. Many more in the Standard Model.

	However, the problem of quadratic divergencies plagues some portion of them:
\[
		\wt{c}_{Q,1}^\mu \cdot
	\ov{Q}\, \gamma^\lambda \wt{F}_{\mu\lambda}\, Q 
	~~\longrightarrow~~
	\Lambda^2 \, a^\mu \cdot \ov{Q}\, \gamma_\mu\, Q~
\]

\end{slide}


%%%%%%%%%%%%%%%%%%%%%%%%%%%%%%  SLIDE %%%%%%%%%%%%%%%%%%%%%%%%%%%%%%%%%%%%%%%
\begin{slide}{Major grouping of operators}

	We group all operators in the Standard Model into 3 types:

\vspace{0.5cm}
\begin{itemize}
\item	{\myit Unprotected}
\[
		\wt{c}_{Q}^\mu \cdot
	\ov{Q}\, \gamma^\lambda \wt{F}_{\mu\lambda}\, Q 
\]
%\vspace{0.5cm}

\item	{\myit UV-enhanced}
\[
	E^2 ~=~ m^2 ~+~ p^2 ~+~ \frac{\eta}{M} \cdot p^3 
\]
%\vspace{0.5cm}

\item	{\myit Soft LV interactions}
\[
	c_{Q}^\mu \cdot
	\ov{Q}\, \gamma^\lambda F_{\mu\lambda}\, Q
\]

\end{itemize}

\end{slide}

%%%%%%%%%%%%%%%%%%%%%%%%%%%%%%  SLIDE %%%%%%%%%%%%%%%%%%%%%%%%%%%%%%%%%%%%%%%
%\begin{slide}{``Protecting'' the operators}
\begin{slide}{Protecting the operators}

        Some symmetry-based properties of dimension five operators forbid
        their transmutation into lower-dimensional ones.
	
\begin{itemize}
\item	{\myit Tensor Structure}. In the Standard Model there are no
	CPT-odd dimension three operators of rank higher than one
\[
	D_{Q}^{\mu\nu\rho} \cdot
	\ov{Q}\, \gamma_{(\mu} F_{\rho)\nu}\, Q
\]

\item	{\myit T-invariance}. In the Standard Model one needs multiple
	loops to flip $ T $-parity of an operator
\[
	c_{Q}^\mu \cdot
	\ov{Q}\, \gamma^\lambda F_{\mu\lambda}\, Q
	~~\nrightarrow~~
	\Lambda^2 \, a^\mu \cdot \ov{Q}\, \gamma_\mu\, Q~
\]

\end{itemize}

\end{slide}


%%%%%%%%%%%%%%%%%%%%%%%%%%%%%%  SLIDE %%%%%%%%%%%%%%%%%%%%%%%%%%%%%%%%%%%%%%%
%\begin{slide}{``Protecting'' the operators}
\begin{slide}{Protecting the operators}

\begin{itemize}
\item 	{\myit Supersymmetry}. 
	In the MSSM, dimension three LV operators do not exist at all.
\begin{align*}
	\wt{c}_{\rm SUSY,Q}^\mu \cdot 
	&
	\left( 
	Y_Q g'\, \ov{Q} \gamma^\lambda \wt{F}_{\mu\lambda} Q   ~~+~~
	g\,  \ov{Q} \gamma^\lambda \wt{W}_{\mu\lambda} Q   ~~+~~
	\right . 
	\\
	& 
	\left .
	g_3\, \ov{Q} \gamma^\lambda \wt{G}_{\mu\lambda} Q 
	\right)  ~~+~~
	\ldots~
\end{align*}
	As usual, supersymmetry turns quadratic divergencies into
	logarithmic.

\item	{\myit Lepton number violation}.
	There is a unique example of an operator of dimension five
	which violates the lepton number by $ \Delta L = 2 $
\[
	\varsigma^{\mu\nu} \cdot
	\left( H^\dag L \right)^T \sigma_{\mu\nu} \left( H^\dag L \right)
\]

\end{itemize}

\end{slide}


%%%%%%%%%%%%%%%%%%%%%%%%%%%%%%  SLIDE %%%%%%%%%%%%%%%%%%%%%%%%%%%%%%%%%%%%%%%
\begin{slide}{Sources of Constraints}

\begin{itemize}

\item {\myit Unprotected Operators}.
	Constraints on dimension 3 operators transfer into a limit
	$ \ll 10^{-31}~\GeV^{-1} $ for unprotected operators.

\item {\myit Ultra-high Energy Cosmic Rays}.
	The fact of observation of high-energy cosmic rays sets typical
	constraints on {\myit UV-enhanced} LV of the order of
	$ 10^{-33-34}~\GeV^{-1} $. 

\item {\myit Precision Experiments}.
	Constraints from Cosmic Rays are not applicable.
	{\myit Soft LV interactions} inducing coupling of nuclear spin to
	the preferred direction put constraints of order of
	$ 10^{-30-31}~\GeV^{-1} $.
	
\end{itemize}

\end{slide}

%%%%%%%%%%%%%%%%%%%%%%%%%%%%%%  SLIDE %%%%%%%%%%%%%%%%%%%%%%%%%%%%%%%%%%%%%%%
\begin{slide}{Sources of Constraints}

\begin{itemize}
\item {\myit Electric Dipole Moments}.
	The operators of the type 
$ 	D_q^{\mu\nu\rho} \cdot 
	\ov{q}\, \gamma_{(\mu} F_{\rho)\nu}\gamma^5\, q  $
	written in terms of effective Hamiltonian possess the signature
	of Electric Dipole Moment interactions. 
	The existing limits on EDMs translate into
	the bound of $ 10^{-12}~\GeV^{-1} $.

\item {\myit Neutrino Phenomenology}.
	At low energies
\[
 	\varsigma^{\mu\nu} \cdot
	\left( H^\dag L \right)^T \sigma_{\mu\nu} \left( H^\dag L \right)
	~~\rightarrow~~
	v^2 \cdot \varsigma^{\mu\nu}_{(\nu)} \cdot \nu^T \sigma_{\mu\nu} \nu 
\]
	It can change the patterns of neutrino oscillations.
	Constraints from reactor and atmospheric neutrino oscillation
	data can be used to limit various components of 
	$ \varsigma^{\mu\nu} $ at $ 10^{-23-24}~\GeV^{-1} $ level.

\end{itemize}

\end{slide}

%%%%%%%%%%%%%%%%%%%%%%%%%%%%%%  SLIDE %%%%%%%%%%%%%%%%%%%%%%%%%%%%%%%%%%%%%%%
\begin{slide}{}

	\vspace{1.5cm}
\usefont{T1}{ppl}{m}{n}\fontsize{30.0pt}{30.0pt}\selectfont
\begin{center}
	Part II\\
	$ CPT $-odd Leptogenesis
\end{center}

\end{slide}

%%%%%%%%%%%%%%%%%%%%%%%%%%%%%%  SLIDE %%%%%%%%%%%%%%%%%%%%%%%%%%%%%%%%%%%%%%%
\begin{slide}{}

	The idea of LV-driven generation of Baryonic Asymmetry of
	the Universe (BAU) resides on the fact that dimension five
	LV operators are odd under $ CPT $

\[
	E^2 ~=~ m^2 ~+~ p^2 ~\pm~ \frac{\eta}{M} \cdot p^3 
\]

	This creates an effective chemical potential of opposite sign
	for quarks and antiquarks
\[
	f_{q,\ov{q}} ~~=~~
	\left\lgroup  1 ~+~  e^{ (E ~\pm~ \mu_{\rm eff}) }/T 
	\right\rgroup^{-1}~.
\]

\end{slide}

%%%%%%%%%%%%%%%%%%%%%%%%%%%%%%  SLIDE %%%%%%%%%%%%%%%%%%%%%%%%%%%%%%%%%%%%%%%
\begin{slide}{}

	Therefore, non-zero BAU is created already in 
	{\myit equilibrium}.

	The amount of final asymmetry is bound to the temperature
	of freeze-out of sphalerons $ T \sim M_W $:
\[
	Y_b  ~=~ \frac{\Delta b}{s}  \sim  \frac{M_W} {\Lambda_{CPT}}
\]
	In such scenarios, however, $ \Lambda_{CPT} $ has to be less
	than $ 10^{12}~\GeV $.

	However, experimental data require $ \Lambda_{CPT} $ to be 
	greater than the {\mybf Planck scale}.

\end{slide}

%%%%%%%%%%%%%%%%%%%%%%%%%%%%%%  SLIDE %%%%%%%%%%%%%%%%%%%%%%%%%%%%%%%%%%%%%%%
\begin{slide}{}

	In  $ CPT $-odd Leptogenesis, we introduce 
\[
	\eta_\text{lepton}^{\mu\nu\rho} \cdot
	\ov{\psi} \gamma_\mu \md_\nu \md_\rho \psi~,
\]
	in the lepton sector.

	{\myit Heavy majorana neutrinos} induce
\[
	\mathcal{L}_{\rm eff} ~~=~~ \frac{Y_{ij}^\nu}{2\, M_R} \, H^\dag L_i^\alpha H^\dag L_{j\alpha}~
+{\rm h.c.},
\]
	An initial $ B - L $ is generated
	through majorana neutrinos at high energies and is preserved
	by sphalerons from $ 10^{12} $ to $ 10^{2}~\GeV $

	Heavy neutrinos keep the lepton number in equilibrium
	(which is non-zero) {\myit until} 
	the neutrino-mediated processes ({\myit L-processes})
	become slower than the Hubble rate.

\end{slide}

%%%%%%%%%%%%%%%%%%%%%%%%%%%%%%  SLIDE %%%%%%%%%%%%%%%%%%%%%%%%%%%%%%%%%%%%%%%
\begin{slide}{}
	Therefore, the final asymmetry is determined by the freeze-out
	temperature of L-violating processes
\[
	\Gamma_L ~~\propto~~ \frac{T^3}{M_R^2} 
	~~\sim~~ \Gamma_H ~~\propto~~ \frac{T^2}{\Mpl},
\]
	which results in
\[
	T_R ~~\propto~~ \frac{M_R^2}{\Mpl}~
	~~\Rrightarrow~~
	T_R/M_W ~\sim~ 10^9 \qquad \text{for}\quad M_R \sim 10^{15}~\GeV
\]

	Another feature of $ CPT $-odd leptogenesis is that one only
	needs {\mybf one generation} of heavy neutrinos.

\end{slide}

%%%%%%%%%%%%%%%%%%%%%%%%%%%%%%  SLIDE %%%%%%%%%%%%%%%%%%%%%%%%%%%%%%%%%%%%%%%
\begin{slide}{}

	The prediction on the BAU we have made was based on dimensional
	counting.

\vspace{0.2cm}
	A more sophisticated prediction should involve the dynamics of
	the number density of leptons and baryons.

\vspace{0.2cm}
	In particular, one has to account for the sphaleron processes
	which change the baryon and lepton densities, leaving intact
	their difference $ n_b - n_l $:

\[
		\p_t\, (n_b + n_l)
	~~~=~~~ -\, \Gsph 
		\lgr   n_b - n_b^\eq \;~+~\;
		       n_l - n_l^\eq 
		\rgr~
\]

	This is a reasonable approximation for dynamics of  $ B + L $.

\[
	\Gsph ~~\simeq~~ \omega\, T~, \qquad\qquad 
	{\rm ~with~}
	\omega ~\simeq~ \,10^{-5}~.
\]

\end{slide}

%%%%%%%%%%%%%%%%%%%%%%%%%%%%%%  SLIDE %%%%%%%%%%%%%%%%%%%%%%%%%%%%%%%%%%%%%%%
\begin{slide}{L-processes}

	The kinetic equations are based on the rate of heavy
	neutrino-mediated processes.
\[
	\mathcal{L}_m  ~~=~~ 
	-\,\frac 12\, M_R\, \left( NN ~+~ \ov{N}\ov{N} \right) ~~+~~
				h_a\cdot \ov{L}_a\ov{ N}H ~~+~~  
				h_a^\dagger\cdot H^\dagger N L_a~
\]	

	Neutrinos can be integrated out to produce an effective
	vertex
\[
	\mathcal{L}_{\rm eff} ~~=~~ \frac{Y_{ij}^\nu}{2\, M_R} \, H^\dag L_i^\alpha H^\dag L_{j\alpha}~
+{\rm h.c.}
\]

%\begin{figure}
\begin{center}
\includegraphics[width=5cm]{lflip.eps}
%\caption{$\Delta L=2$ processes generated by the effective vertex \eqref{L_eff}.}
%\label{lflip_fig}
\end{center}
%\end{figure}


\end{slide}

%%%%%%%%%%%%%%%%%%%%%%%%%%%%%%  SLIDE %%%%%%%%%%%%%%%%%%%%%%%%%%%%%%%%%%%%%%%
\begin{slide}{Boltzmann equations}

	Combining the Hubble expansion rate, and the rate of 
	L-processes and sphaleron processes, one obtains
\[
\begin{split}
% first line
	g_l \frac{d}{dT}Y_l 
	& ~~\;=~~\;
	\frac{0.6}{g_*^{1/2}}\, 
	\frac{\omega\,\Mpl}{T^2}\,
	\lgr g_l (\,Y_l + 12\,\eta_l\,T\,) ~~+~~ 
	     g_b (\,Y_b + 12\,\eta_b\,T\,)  \rgr 
	\;~~%+
	\\
% second line
	& \;~~+~~\;  
	\frac{0.6\,\pi^2}{g_*^{1/2}}\, 
	\frac{\gamma\,\Mpl}{M_R^2}\,
	\cdot (\, Y_l + 12\,\eta_l\,T\,)\\
% third line
	g_b \frac{d}{dT}Y_b 
	& ~~\;=~~\;
	\frac{0.6}{g_*^{1/2}}\, 
	\frac{\omega\,\Mpl}{T^2}\,
	\lgr g_l (\,Y_l + 12\,\eta_l\,T\,) ~~+~~ 
	     g_b (\,Y_b + 12\,\eta_b\,T\,)  \rgr .
\end{split}
\]
	Here  $ \omega $ parametrizes the sphaleron rate, and
	$ \gamma $ the rate of L-processes.

\end{slide}

%%%%%%%%%%%%%%%%%%%%%%%%%%%%%%  SLIDE %%%%%%%%%%%%%%%%%%%%%%%%%%%%%%%%%%%%%%%
\begin{slide}{Soluable limit}

	This system can be solved in the limit of small
	sphaleron rate, $ \omega \to 0 $:

\[
	\frac{d}{dT}Y_l ~~=~~ 
	g_l\,
	\frac{0.6\,\pi^2}{g_*^{1/2}}\, 
	\frac{\gamma\,\Mpl}{M_R^2}\,
	\cdot \lgr\, Y_l ~+~ 12\,\eta_l\,T\, \rgr
\]
	yielding an estimate for the freeze-out temperature
\[
	T_R ~~\sim~~ 
	\frac { M_R^2} { \lambda\, \Mpl }
	~~=~~ 10^{12-14}~\GeV
\]
	
\end{slide}

%%%%%%%%%%%%%%%%%%%%%%%%%%%%%%  SLIDE %%%%%%%%%%%%%%%%%%%%%%%%%%%%%%%%%%%%%%%
\begin{slide}{Evolution}

	We obtain the amount of LV (in terms of $ \eta_l $, $ \eta_b $)
	required to produce the observed BAU, and confront them
	with the known experimental limits on LV.

	\vspace{0.2cm}
	Since the kinetic equations are linear in the number densities,
	it can be done separately for $ \eta_l $ and $ \eta_b $.

	\vspace{0.2cm}
	The kinetic equations depend on an ``effective'' mass of the
	light neutrinos  (through the rate of L-processes)
\[	
	\Gamma_L ~~\sim~~ (\meff)^2\;,
	\quad
	\meff \equiv \left( \sum m_{\nu_i}^2 \right)^{1/2} 
	~~=~~
	0.05~\eV~~\dots~~0.65~\eV
\]

	The lower bound for this parameter comes from the largest
	observed neutrino oscillations;
	the upper bound comes from the cosmological limit on the sum
	of neutrino masses $ \sum m_{\nu_i} $

\end{slide}

%%%%%%%%%%%%%%%%%%%%%%%%%%%%%%  SLIDE %%%%%%%%%%%%%%%%%%%%%%%%%%%%%%%%%%%%%%%
\begin{slide}{Lepton number evolution, $ \eta_l \neq 0 $}

\begin{center}
\includegraphics[width=6cm,angle=270]{l_dom_asymm_bau.ps}
\end{center}
	With the increase of the effective neutrino mass, 
	the rate of L-processes increases, and the lepton number
	density follows closer to the equilibrium curve.

\end{slide}

%%%%%%%%%%%%%%%%%%%%%%%%%%%%%%  SLIDE %%%%%%%%%%%%%%%%%%%%%%%%%%%%%%%%%%%%%%%
\begin{slide}{Baryon number evolution, $ \eta_b \neq 0 $}

\begin{center}
\includegraphics[width=6cm,angle=270]{b_dom_asymm_bau.ps}
\end{center}

	The baryon density curve starts from zero and
	jumps over the equilibrium curve.

\end{slide}

%%%%%%%%%%%%%%%%%%%%%%%%%%%%%%  SLIDE %%%%%%%%%%%%%%%%%%%%%%%%%%%%%%%%%%%%%%%
\begin{slide}{Required amount of LV}

\begin{center}
\includegraphics[width=6cm,angle=270]{scan_log.ps}
\end{center}

$ \eta_l $ changes over the ``physical'' interval.

\end{slide}

%%%%%%%%%%%%%%%%%%%%%%%%%%%%%%  SLIDE %%%%%%%%%%%%%%%%%%%%%%%%%%%%%%%%%%%%%%%
\begin{slide}{Experimental constraints}

	Required amount of Lorentz violation is
\[
	\eta_l ~~\simeq~~ 10^{-22-24}~\GeV^{-1} \qquad
	\text{or} \qquad \eta_b ~~\simeq~~  - 10^{-23}~\GeV^{-1}
\]

%\vspace{0.2cm}
	Low-energy precision experiments set the bounds
\[
|\eta_d-\eta_Q - 0.5(\eta_u-\eta_Q)| ~<~ 10^{-27}~{\rm GeV}^{-1}
\]

	The strongest constraints on the lepton sector come from
	the observation of high-energy cosmic rays, 
	$ E_{max} \sim 10^{12}~\GeV $.

	Non-zero $ \eta_l $ makes possible the decay of the proton
	$ p \to p l \ov{l} $ and prevents acceleration of the rays.

	Observation of these cosmic rays places a bound
\[
	|\eta_L|, ~|\eta_E| ~<~ 10^{-33}~{\rm GeV}^{-1}
%	\qquad \text{vs required} \quad 10^{-22-23}~{\rm GeV}^{-1}
\]

\end{slide}

%%%%%%%%%%%%%%%%%%%%%%%%%%%%%%  SLIDE %%%%%%%%%%%%%%%%%%%%%%%%%%%%%%%%%%%%%%%
\begin{slide}{}

	There is a loophole which the cosmic rays do not cover.
\vspace{0.2cm}

	In the case that all LV is concentrated in the up-quark
	sector, protons will be favored to decay into 
	$ p \to \Delta^{++} \pi^- $.
\vspace{0.2cm}

	This needs the negative sign of $ \eta_U $.
\vspace{0.2cm}

	Then the cosmic rays could exist in the form $ \Delta^{++} $
	and not constrain $ \eta_U $.
\vspace{0.2cm}

	However $ 10^{-22-23}~\GeV^{-1} $ is not compatible 
	with the low energy experiments.

\end{slide}

%%%%%%%%%%%%%%%%%%%%%%%%%%%%%%  SLIDE %%%%%%%%%%%%%%%%%%%%%%%%%%%%%%%%%%%%%%%
\begin{slide}{Higher-dimensional operators}

	Asymmetry generated by dimension seven and so on operators
	can be estimated by the freeze-out temperature
\[
	\eta^{(7)} T_R^3 ~,\qquad \eta^{(9)} T_R^5~,\qquad \dots
\]
	Astrophysical constraints are still strong, as their strength
	scales as $ E_{max}/ \Lambda_{CPT} $.

	However, the same loophole with stable $ \Delta^{++} $ exists,
	and the right-handed up-quark $ CPT $-violation
\[
\eta^{(7)}_U = - [(10^{17}-10^{18})~ {\rm GeV}]^{-3}
\]
	results in the right magnitude of BAU while avoiding experimental
	constraints.

\end{slide}

%%%%%%%%%%%%%%%%%%%%%%%%%%%%%%  SLIDE %%%%%%%%%%%%%%%%%%%%%%%%%%%%%%%%%%%%%%%
\begin{slide}{Conclusions}

\begin{itemize}
\item	$ CPT $-odd Baryogenesis is an equilibrium alternative to the 
	conventional baryogenesis 

\item   $ CPT $-odd Leptogenesis is advantageous in the {\mybf gain}
	of the final BAU --- determined by high freeze-out temperature

\item	Study of dynamics of the number densities in this scenario
	yields the amount of Lorentz violation capable of 
	producing the observed BAU, $ \eta \sim 10^{-22-23}~\GeV^{-1} $

\item   Strong experimental constraints on LV most certainly exclude 
	this type of scenarios

\item	We have only considered the {\myit UV-enhanced} operators;
	other interactions {\myit e.g.} $ \ov{Q}\, \gamma_\mu Q H^\dag H $
	could produce an effective chemical potential, although
	they have problems of their own.
\end{itemize}

\end{slide}



%%%%%%%%%%%%%%%%%%%%%%%%%%%%%%%%  SLIDE %%%%%%%%%%%%%%%%%%%%%%%%%%%%%%%%%%%%%%%
%%\overlays{2}{
%%\begin{slide}{
%%%Lorentz-violating Operators Inducing EDMs
%%}
%%\vspace{-1.5cm}
%%\begin{itemize}
%%\item
%%	We investigate Lorentz-violating (LV) physics at the level of
%%	dimension {\myit five} operators
%%\item
%%	In the Standard Model at 1 GeV there is one vector LV operator inducing EDM
%%\[
%%\mathcolor
%%{\cal L}_{\rm CPT}^{\rm vector} = 
%%	\sum_{i=u,d,s} d^\mu_i \cdot \bar {q}_i \gamma^\lambda \gamma^5 
%%F_{\lambda\mu} q_i
%%\]
%%\untilSlide*{1}
%%{
%%	analogous operators for leptons vanish on the equations of motion.
%%\item
%%	and a number of tensor operators with EDM-like signatures
%%{
%%\mathcolor
%%\begin{multline}
%%\notag
%%{\cal L}_{\rm CPT}^{\rm tensor}=
%%\sum_{i=u,d,s,e,\mu}
%%%D^{\mu\nu\rho}_q \cdot \overline{\psi}_i\gamma_{(\mu} F_{\rho)\nu} \gamma^5 \psi_i
%%D^{\mu\nu\rho}_q \cdot \overline{\psi}_i\gamma_{\mu} F_{\nu\rho} \gamma^5 \psi_i
%%	~~+
%%% tenth line
%%	\\
%%\notag
%%	+\sum_{i=u,d,s}
%%	\widetilde{D}^{\mu\nu\rho} \cdot \overline{\psi}_i
%%	\gamma_{\mu}  G^a_{\nu\rho} t^a \gamma^5\psi_i
%%\end{multline}
%%}
%%}
%%\fromSlide*{2}
%%{
%%\vspace{-0.5cm}
%%\item   
%%	The important property: they {\myit preserve chirality}.
%%	Thus, unlike ordinary EDMs, {\myit decouple linearly} with the scale
%%	of New Physics:
%%\[
%%\mathcolor
%%	d^{\mu}_{\rm CPT} \propto \Lambda_{\rm CPT}^{-1}~{\dblue ,} 
%%	\quad{\rm\dblue whereas}\quad d_{\rm CP} \propto m_q \Lambda_{\rm CP}^{-2}
%%\]	     
%%}
%%\end{itemize}
%%
%%\end{slide}
%%}
%%
%%%%%%%%%%%%%%%%%%%%%%%%%%%%%%%%  SLIDE %%%%%%%%%%%%%%%%%%%%%%%%%%%%%%%%%%%%%%%
%%\begin{slide}[Glitter]{
%%%Lorentz-violating Operators Inducing EDMs
%%}
%%\vspace{-1.5cm}
%%	
%%\begin{itemize}
%%\item
%%	Vector LV operators in the quark sector induce EDM of the 
%%	{\red neutron}
%%\[
%%\mathcolor
%%	d_n = 0.8 d^0_d - 0.4 d^0_u - 0.1 d^0_s
%%\]
%%	where the current experimental constraint 
%%{\mathcolor $|d_n| < 6\times 10^{-26} e{\rm cm} $}
%%	can be applied.
%%
%%\item
%%	EDMs of {\red diamagnetic atoms} have a suppression due to preserved chirality
%%	---
%%	the Schiff moment is induced by the EDMs of nucleons and 
%%	{\myit not} by the $ \pi $-exchange.
%%\item
%%	By the same reason, EDM of a {\red deutron} is suppressed as
%%	{\mathcolor $ \alpha m_q / \Lambda_{\rm QCD} $}.
%%\item
%%	This way, a {\myit\blue discretion between $ CPT $-odd and $ CP $-odd EDMs can
%%	be done}.
%%%\item   
%%%	Tensor LV operators induce more complicated interactions of the 
%%%	nuclear/atomic spin
%%%\[
%%%	H = - \mu {\bf B} \cdot \fr{\bf{S}}{S} - {\cal D}^{ij} E_i \cdot \fr{S_j}{S}
%%%\]
%%%	resulting in an EDM signature $ {\cal D}_{ij} E^i B^j $ with
%%%	12 or 24 hour modulation.
%%\end{itemize}
%%
%%\end{slide}
%%
%%%%%%%%%%%%%%%%%%%%%%%%%%%%%%%%  SLIDE %%%%%%%%%%%%%%%%%%%%%%%%%%%%%%%%%%%%%%%
%%\begin{slide}[Dissolve]{
%%%Lorentz-violating Operators Inducing EDMs
%%}
%%\vspace{-1.5cm}
%%	
%%\begin{itemize}
%%\item
%%	Tensor operators at the Quantum Mechanical level are represented as
%%{\mathcolor
%%\[
%%	H = - \mu {\bf B} \cdot \fr{\bf{S}}{S} - {\cal D}^{ij} E_i \cdot \fr{S_j}{S}
%%\]
%%}
%%	which further produces 
%%	{\mathcolor $ E_i B_k {\cal D}^{ik} $} --- an EDM-like signature.
%%	One expects a 12/24 hour modulation due to rotation of the Earth.
%%\item
%%	In {\red paramagnetic atoms}, atomic EDM occurs due to mixing between
%%	electron levels of opposite parity. 
%%	The resulting EDM is enhanced w.r.t. electron EDM:
%%{\mathcolor
%%$
%%	d({\rm Cs})/d({\rm e}) = 7
%%$
%%}, thus $\mathcolor | {\cal D}_{ik} | ~<~ 10^{-25} e{\rm cm} $~. 
%%
%%
%%\end{itemize}
%%
%%\end{slide}
%%
%%%%%%%%%%%%%%%%%%%%%%%%%%%%%%%%  SLIDE %%%%%%%%%%%%%%%%%%%%%%%%%%%%%%%%%%%%%%%
%%\begin{slide}[Box]{
%%%Lorentz-violating Operators Inducing EDMs
%%}
%%%\vspace{-1.5cm}
%%	
%%\begin{itemize}
%%\item
%%	The most stringent constraints exist on dimension {\myit three}
%%	LV operators, {\myit e.g.}
%%\[\mathcolor
%%	b^\mu \cdot \ov{\psi} \gamma_\mu \gamma_5 \psi
%%\]
%%\item
%%	Dimension five interactions are hard to propagate into dimension
%%	three ones as this involves $ CP $-violation. Nevertheless,
%%{\mathcolor
%%\[
%%	\Lambda_{\rm CPT} ~\sim~ 10^{11-13}~{\rm GeV}
%%\]
%%}
%%	
%%\end{itemize}
%%
%%\end{slide}
%%
%%%%%%%%%%%%%%%%%%%%%%%%%%%%%%%%  SLIDE %%%%%%%%%%%%%%%%%%%%%%%%%%%%%%%%%%%%%%%
%%\begin{slide}[Wipe]{Conclusions}
%%
%%\blue
%%%\myitem{1}{\includegraphics[width=.4cm]{green-bullet-on-white.ps}}
%%\begin{itemize}
%%\item[{\includegraphics[width=.4cm]{green-bullet-on-white.ps}}]
%%	$ CPT $-odd EDMs provide a way of probing New Physics governing
%%	at high energies
%%\vspace{0.3cm}
%%\item[{\includegraphics[width=.4cm]{green-bullet-on-white.ps}}]
%%	Clock-comparison and EDM-searching experiments allow for putting
%%	constraints on parameters of the CPT-breaking theory
%%\vspace{0.3cm}
%%\item[{\includegraphics[width=.4cm]{green-bullet-on-white.ps}}]
%%	Suppression of $ CPT $-odd EDMs in certain cases provides a way
%%	of distinguishing $ CP $-odd EDMs from the latter ones.
%%\end{itemize}	
%%
%%\end{slide}
%%
%%
%%%%%%%%%%%%%%%%%%%%%%%%%%%%%%%%%%  SLIDE %%%%%%%%%%%%%%%%%%%%%%%%%%%%%%%%%%%%%%%
%%%%\begin{slide}{
%%%%%Lorentz-violating Operators Inducing EDMs
%%%%}
%%%%\vspace{-1.5cm}
%%%%	
%%%%\begin{itemize}
%%%%\item
%%%%	Vector LV operators in the quark sector induce EDMs of the neutron
%%%%\[
%%%%	d_n = 0.8 d^0_d - 0.4 d^0_u - 0.1 d^0_s
%%%%\]
%%%%	where the current experimental constraint 
%%%%$|d_n| < 6\times 10^{-26} e{\rm cm} $
%%%%	can be applied.
%%%%\item   
%%%%	Tensor LV operators induce more complicated interactions of the 
%%%%	nuclear/atomic spin
%%%%\[
%%%%	H = - \mu {\bf B} \cdot \fr{\bf{S}}{S} - {\cal D}^{ij} E_i \cdot \fr{S_j}{S}
%%%%\]
%%%%	resulting in an EDM signature $ {\cal D}_{ij} E^i B^j $ with
%%%%	12 or 24 hour modulation.
%%%%\end{itemize}
%%%%
%%%%\end{slide}
%%%%
%%%%%%%%%%%%%%%%%%%%%%%%%%%%%%%%%%  SLIDE %%%%%%%%%%%%%%%%%%%%%%%%%%%%%%%%%%%%%%%
%%%%\begin{slide}{
%%%%%Lorentz-violating Operators Inducing EDMs
%%%%}
%%%%\vspace{-1.5cm}
%%%%	
%%%%\begin{itemize}
%%%%\item
%%%%	In paramagnetic atoms, electron EDM can be significantly enhanced,
%%%%	e.g. for Cesium
%%%%\[
%%%%	\frac{d_{CPT}({\rm Cs})} {d_{CPT}(e)} ~=~ 7~.
%%%%\]
%%%%	which translates into the limit
%%%%\[
%%%%	| {\cal D}_{ik} | ~<~ 10^{-25} e{\rm cm}~.
%%%%\]
%%%%
%%%%\item	
%%%%	Neutron bottles
%%%%\end{itemize}
%%%%\end{slide}
%%%%
%%%%
%%%%%%%%%%%%%%%%%%%%%%%%%%%%%%%%  SLIDE %%%%%%%%%%%%%%%%%%%%%%%%%%%%%%%%%%%%%%%
%%%%
%%%%\overlays{6}{
%%%%\begin{slide}{Lorentz Violation}
%%%%        There are many known examples in the history of physics when 
%%%%	a symmetry of nature, which was assumed to be exact, has fallen under
%%%%	experimental scrutiny.
%%%%
%%%%	Lorentz symmetry is used as a crucial ingredient in the construction
%%%%	of fundamental theories of nature.
%%%%
%%%%	There is a number of seemingly unrelated motives to study Lorentz
%%%%	violating theories:
%%%%\begin{itemize}
%%%%\addSlide{1}{2}{
%%%%\item Time evolution of dark energy (quintessence) 
%%%%      creates a preferred frame which could be detected as a LV 
%%%% 	background, provided that it couples to the Standard Model.
%%%%}
%%%%\addSlide{2}{3}{
%%%%\item Low-energy limits of string theory contain
%%%%      a number of gauge fields with
%%%%        condensed fieldstrengths (fluxes) which
%%%%%(nearly) massless field, some of which
%%%%      carry open Lorentz indices.
%%%%}
%%%%\addSlide{3}{4}{
%%%%\item There are some conjectures that a theory of Quantum Gravity
%%%% 	can manifest itself at low energies through LV modifications
%%%%	of particle dispersion relations.
%%%%}
%%%%\end{itemize}
%%%%
%%%%\untilSlide*{4}
%%%%{\PaleGray
%%%%	Direct experimental constraints on modifications of dispersion
%%%%	relations come from {\myit astrophysical processes}
%%%%	and terrestrial 
%%%%	{\myit clock comparison} and 
%%%%	{\myit torsion balance} experiments.
%%%%}
%%%%\onlySlide*{5}
%%%%{
%%%%	Direct experimental constraints on modifications of dispersion
%%%%	relations come from {\myit astrophysical processes}
%%%%	{\PaleGray and terrestrial 
%%%%	{\myit clock comparison} and 
%%%%	{\myit torsion balance} experiments.}
%%%%}
%%%%\fromSlide*{6}
%%%%{
%%%%	Direct experimental constraints on modifications of dispersion
%%%%	relations come from {\myit astrophysical processes}
%%%%	and terrestrial 
%%%%	{\myit clock comparison} and 
%%%%	{\myit torsion balance} experiments.
%%%%}
%%%%%\addSlide{5}{6}{and terrestrial 
%%%%%	{\myit clock comparison} and 
%%%%%	{\myit torsion balance} experiments.
%%%%%	}
%%%%
%%%%%\FromSlide{5}
%%%%%\begin{itemstep}
%%%%%	\item \fromSlide{4} {astrophysical processes}
%%%%%	\item \fromSlide{5} {terrestrial clock comparison}
%%%%%		\fromSlide{6} {and torsion balance experiments}
%%%%%\end{itemstep}
%%%%%	astrophysical processes and terrestrial
%%%%%	clock comparison and torsion balance 
%%%%%	experiments.
%%%%
%%%%\end{slide}
%%%%}
%%%%
%%%%%%%%%%%%%%%%%%%%%%%%%%%%%%%%  SLIDE %%%%%%%%%%%%%%%%%%%%%%%%%%%%%%%%%%%%%%%
%%%%
%%%%\overlays{5}
%%%%{
%%%%\begin{slide}{\small Lorentz Violating EFT}
%%%%%\small
%%%%%\fnt
%%%%\onlySlide*{1}{
%%%%\DefaultTransition{Box}
%%%%}
%%%%\fromSlide*{2}
%%%%{
%%%%\DefaultTransition{Replace}
%%%%}
%%%%
%%%%	One of the most straightforward ways to study the effects of
%%%%	new physics is to parametrize all our ignorance into a small 
%%%%	number of new parameters.\\
%%%%	Lorentz Violation is created at a high UV scale $ M \sim M_{\rm Pl} $,
%%%%	and at lower scales we have an {\myit effective theory}.
%%%%
%%%%\untilSlide*{1}
%%%%{\PaleGray
%%%%	In QED, the generic expansion in terms of the gauge invariant 
%%%%	operators starts at dimension {\PaleRed three}:
%%%%
%%%%\[
%%%%\nonumber
%%%%{\cal L}_{\rm QED}^{(3)} =
%%%%~-~{\PaleGreen a_\mu}\,  \bar \Psi \gamma_\mu \Psi
%%%%~-~ {\PaleGreen b_\mu}\,  \bar \Psi \gamma^\mu \gamma_5 \Psi 
%%%%~-~ \frac{1}{2}{\PaleGreen H_{\mu\nu}}\bar \Psi \sigma^{\mu\nu} \Psi
%%%%~-~ {\PaleGreen k_\mu}\,  
%%%%\epsilon^{\mu\nu\kappa\lambda} A_\nu \partial_\kappa A_\lambda~.
%%%%%\label{LVqed}
%%%%\]
%%%%}
%%%%
%%%%\fromSlide*{2}
%%%%{
%%%%	In QED, the generic expansion in terms of the gauge invariant 
%%%%	operators starts at dimension {\red three}:
%%%%
%%%%\[
%%%%\nonumber
%%%%{\cal L}_{\rm QED}^{(3)} =
%%%%~-~{\dgreen a_\mu}\,  \bar \Psi \gamma_\mu \Psi
%%%%~-~ {\dgreen b_\mu}\,  \bar \Psi \gamma^\mu \gamma_5 \Psi 
%%%%~-~ \frac{1}{2}{\dgreen H_{\mu\nu}}\bar \Psi \sigma^{\mu\nu} \Psi
%%%%~-~ {\dgreen k_\mu}\,  
%%%%\epsilon^{\mu\nu\kappa\lambda} A_\nu \partial_\kappa A_\lambda~.
%%%%%\label{LVqed}
%%%%\]
%%%%}
%%%%
%%%%\untilSlide*{2}
%%%%{\PaleGray
%%%%	Naive power counting:
%%%%\begin{itemize}
%%%%\item[]	$ \mathcal{L}^{(3)} \sim  M $ \quad  --- {\myit enhanced} by a UV scale
%%%%\item[]   $ \mathcal{L}^{(4)} \sim  1 $
%%%%\item[]   $ \mathcal{L}^{(5)} \sim \;1/M $  \quad --- {\myit suppressed} by a UV scale
%%%%\item[]   \quad...
%%%%\end{itemize}
%%%%}
%%%%
%%%%\fromSlide*{3}
%%%%{
%%%%	Naive power counting:
%%%%\begin{itemize}
%%%%\item[]	$ \mathcal{L}^{(3)} \sim  M $ \quad  --- {\myit enhanced} by a UV scale
%%%%\item[]   $ \mathcal{L}^{(4)} \sim  1 $
%%%%\item[]   $ \mathcal{L}^{(5)} \sim \;1/M $  \quad --- {\myit suppressed} by a UV scale
%%%%\item[]   \quad...
%%%%\end{itemize}
%%%%}
%%%%
%%%%
%%%%\untilSlide*{3}
%%%%{\PaleGray
%%%%	Dimension three operators create a problem:
%%%%	from dimensional counting one expects 
%%%%	{\PaleBlue $ a_\mu \sim M n_\mu $}, 
%%%%	where {\PaleBlue $ n_\mu $} is a unit vector, and
%%%%	$ M $ is the scale of New Physics.
%%%%	That creates {\myit\PaleBlue disastrous} effects.
%%%%}
%%%%\fromSlide*{4}
%%%%{
%%%%	Dimension three operators create a problem:
%%%%	from dimensional counting one expects 
%%%%	{\blue $ a_\mu \sim M n_\mu $}, 
%%%%	where {\blue $ n_\mu $} is a unit vector, and
%%%%	$ M $ is the scale of New Physics.
%%%%	That creates {\myit\blue disastrous} effects.
%%%%}
%%%%
%%%%
%%%%\untilSlide*{4}
%%%%{\PaleGray
%%%%	Dimension {\myit five} operators are naturally suppressed.
%%%%	One of the solutions is if such dimension three and four
%%%%	operators are suppressed by a {\PaleRed symmetry}.
%%%%}
%%%%\fromSlide*{5}
%%%%{
%%%%	Dimension {\myit five} operators are naturally suppressed.
%%%%	One of the solutions is if such dimension three and four
%%%%	operators are suppressed by a {\red symmetry}.
%%%%}
%%%%
%%%%\end{slide}
%%%%}
%%%%
%%%%%%%%%%%%%%%%%%%%%%%%%%%%%%%%  SLIDE %%%%%%%%%%%%%%%%%%%%%%%%%%%%%%%%%%%%%%%
%%%%\overlays{5}
%%%%{
%%%%\begin{slide}{\small Supersymmetry \fromSlide{5}{rules!}}
%%%%\onlySlide*{1}{
%%%%\DefaultTransition{Dissolve}
%%%%}
%%%%\fromSlide*{2}
%%%%{
%%%%\DefaultTransition{Replace}
%%%%}
%%%%
%%%%	{\myit Supersymmetry} (SUSY) has been introduced a few 
%%%%	decades ago and is known to have salutary properties for QFT.\\
%%%%	In particular, it removes quadratic divergencies.
%%%%
%%%%	It appears to be beneficial for LV theories too!
%%%%
%%%%\addSlide{1}{2}
%%%%{
%%%%	In particular in the Minimal Supersymmetric Standard Model,
%%%%	LV operators of dimensions three and four are prohibited
%%%%	by the requirements of supersymmetry and gauge invariance.
%%%%
%%%%	Lorentz-violating interactions start only at dimension five.
%%%%}
%%%%
%%%%\untilSlide*{2}
%%%%{\PaleGray
%%%%	Once Supersymmetry is softly broken, dimension three LV
%%%%	operators can be induced, with coefficients controlled
%%%%	by the soft-breaking mass scale, 
%%%%	{\PaleRed $ m_s \sim 1~{\rm TeV} $}.	
%%%%}
%%%%\fromSlide*{3}
%%%%{
%%%%	Once Supersymmetry is softly broken, dimension three LV
%%%%	operators can be induced, with coefficients controlled
%%%%	by the soft-breaking mass scale, 
%%%%	{\red $ m_s \sim 1~{\rm TeV} $}.	
%%%%}
%%%%
%%%%\untilSlide*{3}
%%%%{\PaleGray
%%%%	Dimension three operators are now pure quantum effects, and
%%%%	effectively, the divergencies are stabilized at the 
%%%%	supersymmetric threshold:
%%%%\[
%%%%[LV]_{\rm dim~3} ~~\sim~~ ({\rm loop~factor}) \, 
%%%%\;{\PaleRed m_{s}^2}\;
%%%%\times\; [LV]_{{\rm dim}~5}~.
%%%%\]
%%%%}
%%%%
%%%%\fromSlide*{4}
%%%%{
%%%%	Dimension three operators are now pure quantum effects, and
%%%%	effectively, the divergencies are stabilized at the 
%%%%	supersymmetric threshold:
%%%%\[
%%%%[LV]_{\rm dim~3} ~~\sim~~ ({\rm loop~factor}) \, 
%%%%\;{\red m_{s}^2}\;
%%%%\times\; [LV]_{{\rm dim}~5}~.
%%%%\]
%%%%}
%%%%
%%%%\addSlide{4}{5}
%%%%{
%%%%	$ \Rrightarrow $ This might lead to a solution of the
%%%%	naturalness problem: why the lower dimensional LV operators
%%%%	are so much suppressed as compared to their natural scale.
%%%%}
%%%%
%%%%\end{slide}
%%%%}
%%%%
%%%%%%%%%%%%%%%%%%%%%%%%%%%%%%%%  SLIDE %%%%%%%%%%%%%%%%%%%%%%%%%%%%%%%%%%%%%%%
%%%%\overlays{5}
%%%%{
%%%%\begin{slide}[Replace]{\small Lorentz-violating in SQED at dimension five}
%%%%
%%%%	SQED --- supersymmetric extension of QED --- is the 
%%%%	simplest gauge supersymmetric theory.
%%%%
%%%%\addSlide{1}{2}
%%%%{	
%%%%	At dimension five level there are only {\myit three} LV interactions
%%%%	(whereas in ordinary QED there are about {\myit ten} of them).
%%%%}
%%%%
%%%%\untilSlide*{2}
%%%%{\PaleGray
%%%%	Upon SUSY breaking LV SQED {\myit does} generate dimension 
%%%%	{\myit three} interactions:
%%%%\\[0.5cm]
%%%%
%%%%\begin{center}
%%%%%	\includegraphics[width=2.4cm,height=2.7cm,keepaspectratio]
%%%%%	{diag_chiral_SB_chiral_LV_A_pale.ps} 
%%%%\end{center}
%%%%}
%%%%\fromSlide*{3}
%%%%{
%%%%	Upon SUSY breaking LV SQED {\myit does} generate dimension 
%%%%	{\myit three} interactions:
%%%%\\[0.5cm]
%%%%
%%%%\begin{center}
%%%%%	\includegraphics[width=2.4cm,height=2.7cm,keepaspectratio]
%%%%%	{diag_chiral_SB_chiral_LV_A_blue.ps} 
%%%%\end{center}
%%%%}
%%%%
%%%%\vspace{0.3cm}
%%%%\addSlide{3}{4}
%%%%{
%%%%	Very strigent constraints on the {\myit Standard Model}
%%%%	Lorentz violation come from astrophysical observations, as
%%%%	LV modifies dispersion relations --- this might cause birefridgence
%%%%	of light and vacuum \v{C}erenkov radiation.
%%%%}
%%%%
%%%%\addSlide{4}{5}
%%%%{
%%%%	Supersymmetric LV almost {\myit do not modify} the dispersion
%%%%	relations and so evade these stringent constraints.
%%%%}
%%%%
%%%%\end{slide}
%%%%}
%%%%
%%%%
%%%%%%%%%%%%%%%%%%%%%%%%%%%%%%%%  SLIDE %%%%%%%%%%%%%%%%%%%%%%%%%%%%%%%%%%%%%%%
%%%%
%%%%\overlays{6}
%%%%{
%%%%\begin{slide}{\small Conclusions}
%%%%\onlySlide*{1}{
%%%%\DefaultTransition{Blinds}
%%%%}
%%%%\fromSlide*{2}
%%%%{
%%%%\DefaultTransition{Replace}
%%%%}
%%%%
%%%%\begin{itemize}
%%%%\item		Analysis of low energy effective behavior shows that 
%%%%		LV in SQED leads to detectable effects, being limited
%%%%		at the level of $ 10^{-10} - 10^{-12} $.
%%%%\vspace{0.2cm}
%%%%\addSlide{1}{2}
%%%%{
%%%%\item
%%%%		Thus, theories predicting LV at dimension five level
%%%%		have a potential naturalness problem.
%%%%}
%%%%\vspace{0.2cm}
%%%%\addSlide{2}{3}
%%%%{
%%%%\item
%%%%		This might be healed also by symmetry arguments,
%%%%		{\myit e.g. CPT}, which
%%%%		could {\myit prohibit} dimension five interactions.
%%%%}
%%%%\vspace{0.2cm}
%%%%\addSlide{3}{4}
%%%%{
%%%%\item
%%%%		Dimension {\myit six} LV interactions are CPT-invariant
%%%%		and {\myit not yet} excluded by experiment.
%%%%}
%%%%\vspace{0.2cm}
%%%%\addSlide{4}{5}
%%%%{
%%%%\item
%%%%		The next stage to consider is a
%%%%		{\myit Dimension six LV extension of the Minimal Supersymmetric
%%%%		Standard Model} (MSSM).
%%%%}
%%%%\end{itemize}
%%%%
%%%%\vspace{0.4cm}
%%%%\addSlide{5}{6}
%%%%{
%%%%		P.A.Bolokhov, S.G.Nibbelink and M.Pospelov, 	
%%%%		Phys. Rev. D72:015013 (2005), hep-ph/0505029
%%%%\\[0.1cm]
%%%%
%%%%		Credits: University of Guelph
%%%%		\quad$\bullet$\quad 
%%%%		Perimeter Institute
%%%%		\quad$\bullet$\quad 
%%%%		University of Victoria
%%%%
%%%%\vspace{0.3cm}
%%%%\begin{center}
%%%%\myit
%%%%	August 2005 ~$\bullet$~ CAM 2005 ~$\bullet$~ San Diego, CA
%%%%\end{center}
%%%%}
%%%%
%%%%\end{slide}
%%%%}
%%%%
%%%%%%%%%%%%%%%%%%%%%%%%%%%%%%%%%%  SLIDE %%%%%%%%%%%%%%%%%%%%%%%%%%%%%%%%%%%%%%%
%%%%
%%%%\overlays{5}
%%%%{
%%%%\begin{slide}{Introduction: Why Lorentz Violation?}
%%%%%\vspace{-0.5cm}
%%%%%\small
%%%%%\fnt
%%%%\onlySlide*{1}{
%%%%\DefaultTransition{Glitter}
%%%%}
%%%%\fromSlide*{2}
%%%%{
%%%%\DefaultTransition{Replace}
%%%%}
%%%%
%%%%        There are many known examples in the history of physics when 
%%%%	a symmetry of nature, which was assumed to be exact, has fallen under
%%%%	experimental scrutiny.
%%%%
%%%%	Lorentz symmetry is used as a crucial ingredient in the construction
%%%%	of fundamental theories of nature.
%%%%
%%%%	There is a number of seemingly unrelated motives to study Lorentz
%%%%	violating theories:
%%%%\begin{itemstep}
%%%%\item Time evolution of dark energy (quintessence) 
%%%%      creates a preferred frame which could be detected as a LV 
%%%% 	background, provided that it couples to the Standard Model.
%%%%\item Low-energy limits of string theory contain
%%%%      a number of gauge fields with
%%%%        condensed fieldstrengths (fluxes) which
%%%%%(nearly) massless field, some of which
%%%%      carry open Lorentz indices.
%%%%\item There are some conjectures that a theory of Quantum Gravity
%%%% 	can manifest itself at low energies through LV modifications
%%%%	of particle dispersion relations.
%%%%\end{itemstep}
%%%%
%%%%\fromSlide{4} 
%%%%{
%%%%	Direct experimental constraints on modifications of dispersion
%%%%	relations come from {\myit astrophysical processes} 
%%%%	\fromSlide{5}{and terrestrial 
%%%%	{\myit clock comparison} and 
%%%%	{\myit torsion balance} experiments.
%%%%	}
%%%%}
%%%%
%%%%%\FromSlide{5}
%%%%%\begin{itemstep}
%%%%%	\item \fromSlide{4} {astrophysical processes}
%%%%%	\item \fromSlide{5} {terrestrial clock comparison}
%%%%%		\fromSlide{6} {and torsion balance experiments}
%%%%%\end{itemstep}
%%%%%	astrophysical processes and terrestrial
%%%%%	clock comparison and torsion balance 
%%%%%	experiments.
%%%%\end{slide}
%%%%}
%%%%
%%%%%%%%%%%%%%%%%%%%%%%%%%%%%%%%  SLIDE %%%%%%%%%%%%%%%%%%%%%%%%%%%%%%%%%%%%%%%
%%%%
%%%%\overlays{6}
%%%%{
%%%%\begin{slide}{\small Lorentz Violating EFT}
%%%%%\small
%%%%%\fnt
%%%%\onlySlide*{1}{
%%%%\DefaultTransition{Box}
%%%%}
%%%%\fromSlide*{2}
%%%%{
%%%%\DefaultTransition{Replace}
%%%%}
%%%%
%%%%	One of the most straightforward ways to study the effects of
%%%%	new physics is to parametrize all our ignorance into a small 
%%%%	number of new parameters.
%%%%
%%%%\FromSlide{2}
%%%%	In QED, the generic expansion in terms of the gauge invariant 
%%%%	operators starts at dimension {\red three}:
%%%%
%%%%\[
%%%%\nonumber
%%%%{\cal L}_{\rm QED}^{(3)} =
%%%%~-~{\dgreen a_\mu}\,  \bar \Psi \gamma_\mu \Psi
%%%%~-~ {\dgreen b_\mu}\,  \bar \Psi \gamma^\mu \gamma_5 \Psi 
%%%%~-~ \frac{1}{2}{\dgreen H_{\mu\nu}}\bar \Psi \sigma^{\mu\nu} \Psi
%%%%~-~ {\dgreen k_\mu}\,  
%%%%\epsilon^{\mu\nu\kappa\lambda} A_\nu \partial_\kappa A_\lambda~.
%%%%%\label{LVqed}
%%%%\]
%%%%
%%%%\FromSlide{3}
%%%%	Dimension three operators create a problem:
%%%%	from dimensional counting one expects 
%%%%	{\blue $ a_\mu \sim M n_\mu $}, 
%%%%	where {\blue $ n_\mu $} is a unit vector, and
%%%%	$ M $ is the scale of New Physics.
%%%%	That creates {\myit\blue disastrous} effects.
%%%%
%%%%\FromSlide{4}
%%%%	Even if the LV coefficients are tuned small, they will be 
%%%%	exploded by quantum corrections coming from quadratic divergencies
%%%%	of the higher-dimensional operators:
%%%%\FromSlide{5}
%%%%\[
%%%%[LV]_{\rm dim~3} ~~\sim~~ ({\rm loop~factor}) \, 
%%%%\;\Lambda_{UV}^2\;
%%%%\times\; [LV]_{{\rm dim}~5}~. 
%%%%\]
%%%%
%%%%	Dimension five operators have to be tuned too \quad
%%%%	$ \Longrightarrow $ \quad no room for LV interactions.
%%%%
%%%%\FromSlide{6}
%%%%	One of the solutions is if such quadratic divergencies
%%%%	are suppressed by a {\red symmetry}.
%%%%
%%%%\end{slide}
%%%%}
%%%%
%%%%%%%%%%%%%%%%%%%%%%%%%%%%%%%%  SLIDE %%%%%%%%%%%%%%%%%%%%%%%%%%%%%%%%%%%%%%%
%%%%\overlays{5}
%%%%{
%%%%\begin{slide}[Replace]{\small Supersymmetry \fromSlide{5}{rules!}}
%%%%
%%%%	Recently it has been proposed that SUSY could provide a 
%%%%	powerful selection rule on admissible forms of LV interactions.
%%%%
%%%%\FromSlide{2}
%%%%	In particular in the Minimal Supersymmetric Standard Model,
%%%%	LV operators of dimensions three and four are prohibited
%%%%	by the requirements of supersymmetry and gauge invariance.
%%%%
%%%%	Lorentz-violating interactions start only at dimension five.
%%%%
%%%%\FromSlide{3}
%%%%	Once Supersymmetry is softly broken, dimension three LV
%%%%	operators can be induced, with coefficients controlled
%%%%	by the soft-breaking mass scale, 
%%%%	{\red $ m_s \sim 1~{\rm TeV} $}.	
%%%%
%%%%\FromSlide{4}
%%%%	Dimension three operators are now pure quantum effects, and
%%%%	effectively, the divergencies are stabilized at the 
%%%%	supersymmetric threshold:
%%%%\[
%%%%[LV]_{\rm dim~3} ~~\sim~~ ({\rm loop~factor}) \, 
%%%%\;{\red m_{s}^2}\;
%%%%\times\; [LV]_{{\rm dim}~5}~.
%%%%\]
%%%%\FromSlide{5}
%%%%	$ \Rrightarrow $ This might lead to a solution of the
%%%%	naturalness problem: why the lower dimensional LV operators
%%%%	are so much suppressed as compared to their natural scale.
%%%%
%%%%%%\rnode{left1}{\strut}
%%%%%%\[
%%%%%%\nonumber
%%%%%%{\cal L}_{\rm QED}^{(3)} = 
%%%%%%~-~{\black a_\mu}\,  \bar \Psi \gamma_\mu \Psi
%%%%%%~-~ {\black b_\mu}\,  \bar \Psi \gamma^\mu \gamma_5 \Psi 
%%%%%%~-~ \frac{1}{2}{\black H_{\mu\nu}}\bar \Psi \sigma^{\mu\nu} \Psi
%%%%%%~-~ {\black k_\mu}\,  
%%%%%%\epsilon^{\mu\nu\kappa\lambda} A_\nu \partial_\kappa A_\lambda~.
%%%%%%\]
%%%%%%\rnode{right1}{\strut}
%%%%%%\ncline[linecolor=red]{left1}{right1}
%%%%\end{slide}
%%%%}
%%%%%%%%%%%%%%%%%%%%%%%%%%%%%%%%  SLIDE %%%%%%%%%%%%%%%%%%%%%%%%%%%%%%%%%%%%%%%
%%%%\overlays{10}
%%%%{
%%%%\begin{slide}[Replace]{\small Questions we ask}
%%%%
%%%%\fromSlide{1}{
%%%%	In our work we have set up the following objectives, confining
%%%%	ourselves to SQED, but being on a path to the full 
%%%%	Lorentz-violating MSSM:
%%%%	}
%%%%\begin{itemstep}
%%%%	\item 
%%%%	\fromSlide{2}
%%%%	{prove the absence of the naturalness problem in the LV sector}
%%%%	\item 
%%%%	\fromSlide{3}{
%%%%	derive phenomenological constraints on the LV parameters
%%%%		of SQED}
%%%%\end{itemstep}
%%%%
%%%%\fromSlide{4}{	
%%%%	In detail, the investigation includes:}
%%%%\begin{itemstep}
%%%%	\item 
%%%%	\fromSlide{5}{
%%%%		find out whether D-term anomaly (cancellation) is disturbed
%%%%		by the presence of LV interactions}
%%%%	\item 
%%%%	\fromSlide{6}{
%%%%	prove that gauge anomaly does not arise due to 
%%%%      	      the LV interactions}
%%%%	\item
%%%%	\fromSlide{7}{
%%%%	derive the Renormalization Group (RG) evolution for the LV
%%%%	operators, and show that quadratic divergencies do not 
%%%%	arise in the limit of exact supersymmetry
%%%%	}
%%%%	\item
%%%%	\fromSlide{8}{
%%%%	investigate the consequences of SUSY breaking by introducing
%%%%	soft-breaking masses for super-partners of electrons
%%%%	}
%%%%	\item
%%%%	\fromSlide{9}{
%%%%	one of the most curious questions about dimension three operators 
%%%%	and a subject of long disputes is the Chern-Simons operator;
%%%%	we show that it is not induced by SUSY breaking
%%%%	}
%%%%	\item
%%%%	\fromSlide{10}{
%%%%	study the phenomenological consequences of the soft SUSY breaking;
%%%%	put constraints on the LV parameters
%%%%	}
%%%%\end{itemstep}
%%%%
%%%%\end{slide}
%%%%}
%%%%%%%%%%%%%%%%%%%%%%%%%%%%%%%%  SLIDE %%%%%%%%%%%%%%%%%%%%%%%%%%%%%%%%%%%%%%%
%%%%
%%%%\overlays{11}
%%%%{
%%%%\begin{slide}{\small Lorentz Violation and SQED}
%%%%\onlySlide*{1}{
%%%%\DefaultTransition{Blinds}
%%%%}
%%%%\fromSlide*{2}
%%%%{
%%%%\DefaultTransition{Replace}
%%%%}
%%%%
%%%%\FromSlide{1}
%%%%	Supersymmetric Quantum Electrodynamics is described by two
%%%%	chiral superfields 
%%%%	{\blue $ \Phi_+ $} and 
%%%%	{\blue $ \Phi_- $}, that are
%%%%	oppositely charged under a U(1) gauge superfield 
%%%%	{\blue $ V $}:
%%%%\FromSlide{2}
%%%%%%
%%%%%% The SQED lagrangian
%%%%\begin{eqnarray*}
%%%%% first line
%%%%\mathcal{L}_{\rm SQED} & ~=~
%%%%&
%%%%\int d^4\theta\, \Big\{ 
%%%%   \overline{\Phi}_+ e^{2eV} \Phi_+ ~+~
%%%%   \overline{\Phi}_- e^{-2eV} {\Phi}_-  \Big\} ~+~ \\
%%%%% second line
%%%%%\label{SQED}
%%%%& + &
%%%%\int d^2\theta\, \left\lgroup \frac{1}{4}\,  WW ~+~m_e\, \Phi_-\Phi_+ 
%%%%	\right\rgroup ~+~
%%%%\int d^2\overline{\theta}\, 
%%%%\left\lgroup \frac{1}{4}\, \overline{W}\,\overline{W} ~+~ 
%%%%\overline{m}_e\, \overline{\Phi}_+\overline{\Phi}_- 
%%%%\right\rgroup~. 
%%%%% third line
%%%%\nonumber 
%%%%\end{eqnarray*}
%%%%%
%%%%
%%%%\FromSlide{3}
%%%%	In analogy to the ordinary LV operators, supersymmetric 
%%%%	LV operators can be constructed of the 
%%%%\FromSlide{4}
%%%%	fundamental
%%%%	{\blue super}fields {\blue $ \Phi_+ $} and 
%%%%	{\blue $ \Phi_- $}, 
%%%%\FromSlide{5}
%%%%	and {\blue super}gauge covariant
%%%%	derivatives 
%%%%	{\blue $ \nabla_\alpha $, $ \overline{\nabla}_{\dot\alpha} $}
%%%%\FromSlide{6}
%%%%	and {\blue $ T^{\mu\nu\rho\sigma\dots} $} ---
%%%%	{\myit arbitrary} constant tensor coefficients with
%%%%	Lorentz indices that specify the breakdown of the Lorence
%%%%	invariance.
%%%%
%%%%\FromSlide{7}
%%%%	We require that all LV operators be
%%%%\begin{itemize}
%%%%\FromSlide{8}
%%%%\item supersymmetric 
%%%%\FromSlide{9}
%%%%\item local super gauge invariant with chiral gauge parameters
%%%%\FromSlide{10}
%%%%\item have local component expressions. 
%%%%\end{itemize}
%%%%
%%%%\FromSlide{11}
%%%%	This is enough to rule out the dimension three and four 
%%%%	interactions, in particular the {\myit
%%%%	Chern-Simons operator} --- 
%%%%	{\blue
%%%%	$ k_\mu\,  
%%%%\epsilon^{\mu\nu\kappa\lambda} A_\nu \partial_\kappa A_\lambda $}.
%%%%
%%%%\end{slide}
%%%%}
%%%%
%%%%%%%%%%%%%%%%%%%%%%%%%%%%%%%%  SLIDE %%%%%%%%%%%%%%%%%%%%%%%%%%%%%%%%%%%%%%%
%%%%
%%%%\overlays{5}
%%%%{
%%%%\begin{slide}{ CPT-violating dimension five LV operators }
%%%%\onlySlide*{1}{
%%%%\DefaultTransition{Glitter}
%%%%}
%%%%\fromSlide*{2}
%%%%{
%%%%\DefaultTransition{Replace}
%%%%}
%%%%	
%%%%	It turns out that the requirements of SUSY and gauge invariance
%%%%	greatly reduce the number of LV operators.
%%%%
%%%%	At dimension five level, SQED is compatible with only 3 operators:
%%%%
%%%%\FromSlide{2}
%%%%	A matter operator, parameterized by a vector LV background
%%%%	{\red $ N_\pm^\mu $}:
%%%%%%
%%%%%% the electron and positron operators
%%%%\[
%%%%%\label{LV_matter}
%%%%  \mathcal{L}_{\mathrm{LV}}^{\mathrm{matter}} ~=~ 
%%%%\frac{1}{M}\,   \int d^4\theta \Big\{ 
%%%%% electron
%%%%{\red{N_+^\mu}}\, \overline{\Phi}_+ e^{2eV} i \nabla_\mu \Phi_+ 
%%%%% positron
%%%%~+~ 
%%%%{\red N_{-}^\mu}\, \overline{\Phi}_- e^{-2eV} i \nabla_\mu  {\Phi}_-
%%%%                 \Big\}~, 
%%%%\]
%%%%%
%%%%	
%%%%\FromSlide{3}
%%%%	an operator in the gauge sector, parameterized by a(nother)
%%%%	vector LV background {\red $ N^\mu $}:
%%%%%% photon -- Kahler term
%%%%\[
%%%%%\label{LV_gauge}
%%%%\mathcal{L}_{\mathrm{LV\ dim\ 5}}^{\mathrm{gauge\ (V)}} ~=~ 
%%%%\frac 1M \int d^4\theta \, 
%%%%{\red N^\kappa}\, \overline{W} \bar{\sigma}_\kappa W~,   
%%%%\]
%%%%%
%%%%\FromSlide{4}
%%%%	and another operator of the gauge sector, written as a 
%%%%	superpotential, and parameterized
%%%%	by a {\myit real irreducible 3-rank tensor} {\red $T^{\mu\nu\rho}$}:
%%%%%% photon -- superpotential term
%%%%\[
%%%%%\label{LV_gauge_Tterm}
%%%%\mathcal{L}_{\mathrm{LV\ dim\ 5}}^{\mathrm{gauge\ (T)}} ~=~ 
%%%%\frac 1{4M} 
%%%%\int d^2\theta \, {\red T^{\lambda\, \mu\nu}} \,
%%%%        W \sigma_{\mu\nu} \, \partial_\lambda W  
%%%%~+~ \frac 1{4M} 
%%%%\int d^2\theta \, {\red \overline{T}^{\lambda\, \mu\nu}} \,
%%%%        \overline{W} \,\bar{\sigma}_{\mu\nu}\, \partial_\lambda\overline{W}  
%%%%~.
%%%%\]
%%%%%
%%%%\FromSlide{5}
%%%%	In a non-abelian theory the last operator does not exist.
%%%%\end{slide}
%%%%}
%%%%
%%%%%%%%%%%%%%%%%%%%%%%%%%%%%%%%  SLIDE %%%%%%%%%%%%%%%%%%%%%%%%%%%%%%%%%%%%%%%
%%%%
%%%%\overlays{5}
%%%%{
%%%%\begin{slide}{ CPT-conserving dimension 6 operators }
%%%%\onlySlide*{1}{
%%%%\DefaultTransition{Blinds}
%%%%}
%%%%\fromSlide*{2}
%%%%{
%%%%\DefaultTransition{Replace}
%%%%}
%%%%	At the dimension 6 level, the set of operators is more diverse.
%%%%\FromSlide{2}
%%%%	In the gauge sector, there is only one operator, 
%%%%	{\myit superpotential} operator:
%%%%\begin{gather} 
%%%%\nonumber
%%%%{\cal L}_{\rm{LV\ dim\ 6}}^{\rm{super}} ~=~ \frac{1}{M^2}
%%%%\int d^2 \theta \, 
%%%%%\Big( 
%%%%%A^{\mu\nu} \, \Phi_+ \Phi_-\, W \sigma_{\mu\nu} W ~+~ 
%%%%%B^{\mu\nu} \, W \sigma_{\mu\nu} \Box W ~+~ 
%%%%%C^{\mu\nu} \, W \sigma_{\mu\rho} \partial_\nu \partial^\rho W 
%%%%%\nonumber \\[2ex]
%%%%%~+~ 
%%%%{\red S^{\mu\nu}}\, W \partial_\mu \partial_\nu W 
%%%%%~+~ T^{\mu\nu\, \rho\sigma} \, 
%%%%%W \sigma_{\mu\nu} \partial_\rho \partial_\sigma W 
%%%%%\Big) 
%%%%~+~ \text{h.c.}~,\qquad  {\red S^{\mu\nu}} = {\red S^{\nu\mu}}~. 
%%%%%\label{LV_dim6_Fterm}
%%%%\end{gather}
%%%%%
%%%%\FromSlide{3}
%%%%	However, in the matter sector, there are 3 types LV interactions:
%%%%%
%%%%\begin{gather*}
%%%%{\cal L}_{\rm LV\ dim\ 6}^{\rm matter}  ~=~ \frac{1}{M^2}
%%%%\int d^4 \theta\, \Big[
%%%%\overline{\Phi}_\pm e^{\pm 2eV} \Phi_\pm \, 
%%%%\left\lgroup 
%%%%{\red A_\pm^{\mu\nu}} \, D \sigma_{\mu\nu} W ~+~ 
%%%%{\red \overline{A}_\pm^{\mu\nu}} \, 
%%%%	\overline{D} \bar\sigma_{\mu\nu} \overline{W}
%%%%\right\rgroup 
%%%%\nonumber \\[2ex]
%%%%~+~ 
%%%%{\red S_\pm^{\mu\nu}}\,  \overline{\Phi}_\pm e^{\pm 2eV} 
%%%%\lbrace \nabla_\mu, \nabla_\nu \rbrace \Phi_\pm  
%%%%~+~ 
%%%%{\red Z^{\mu\nu}}\,  \Phi_- \lbrace \nabla_\mu, \nabla_\nu \rbrace \Phi_+ 
%%%%~+~ 
%%%%{\red \overline{Z}^{\mu\nu}}\,  
%%%%\overline{\Phi}_- \lbrace \nabla_\mu, \nabla_\nu \rbrace \overline{\Phi}_+ 
%%%% \Big]~, 
%%%%% \label{LV_dim6_Dterm}
%%%%\end{gather*}
%%%%%
%%%%\FromSlide{4}
%%%%	The terms 
%%%%{\red
%%%%$ F_{\mu\rho}F_{\nu\sigma}F^{\rho\sigma} $}
%%%%	and
%%%%{\red
%%%%$ F_{\rho\sigma}F^{\rho\sigma}F_{\mu\nu} $}
%%%%	have not appeared, although they {\myit do} appear
%%%%	in non-commutative SQED. 
%%%%	However, they are incompatible with conventional ``commuting''
%%%%	supersymmetry.
%%%%
%%%%\FromSlide{5}
%%%%	{\myit We will be interested in dimension 5 operators only. }
%%%%\end{slide}
%%%%}
%%%%
%%%%%%%%%%%%%%%%%%%%%%%%%%%%%%%%  SLIDE %%%%%%%%%%%%%%%%%%%%%%%%%%%%%%%%%%%%%%%
%%%%
%%%%\overlays{5}
%%%%{
%%%%\begin{slide}[Replace]{ Absence of the $D$-term anomaly }
%%%%\vspace{-0.5cm}
%%%%	Generically, supersymmetry is free of dangerous quadratic
%%%%	divergencies.
%%%%\FromSlide{2}
%%%%	However, there is one exception, the $ D $-term anomaly:
%%%%
%%%%\begin{center}
%%%%	\includegraphics[height=1.0cm,keepaspectratio]{diag_FI2.ps}
%%%%\end{center}
%%%%\FromSlide{3}
%%%%	where
%%%%\[
%%%%        V ~=~  -~ \theta\,\sigma^\mu\, \overline{\theta}\, A_\mu ~+~
%%%%                i \theta^2\, \overline{\theta}\, \overline{\lambda} 
%%%%                ~-~
%%%%                i \overline{\theta}{}^2\, \theta\lambda
%%%%                ~+~
%%%%                \frac{1}{2}
%%%%                \theta^2\overline{\theta}{}^2\, D~.
%%%%\]
%%%%\FromSlide{4}
%%%%	If we introduce an ``exact'' LV propagator	
%%%%%
%%%%\[
%%%%\raisebox{-1ex}{\includegraphics[width=1.4cm,keepaspectratio]{diag_resmprop.ps}}
%%%%\quad = \quad 
%%%%\frac 1{\Box}~  \frac 1{1 ~+~  \frac{ N_\pm^\mu} M \, i\partial_\mu}~, 
%%%%\]
%%%%%
%%%%\FromSlide{5}
%%%%	then we come to a cancellation:
%%%%%
%%%%\[
%%%%\raisebox{-3.0ex}{\includegraphics[height=1.0cm,keepaspectratio]{diag_FI1.ps}}
%%%%\quad = \quad 
%%%%\raisebox{-3.0ex}{\includegraphics[height=1.0cm,keepaspectratio]{diag_FI2.ps}}
%%%%\quad = \quad 0~,
%%%%\]
%%%%	which vanishes for non-anomalous theories, where the sum of 
%%%%	charges is zero.
%%%%%
%%%%
%%%%\end{slide}
%%%%}
%%%%
%%%%%%%%%%%%%%%%%%%%%%%%%%%%%%%%  SLIDE %%%%%%%%%%%%%%%%%%%%%%%%%%%%%%%%%%%%%%%
%%%%
%%%%\overlays{5}
%%%%{
%%%%\begin{slide}[Replace]{ $ D $-term at higher loops }
%%%%
%%%%	At two loops, one should consider more complicated situations,
%%%%\FromSlide{2}
%%%%%
%%%%\[
%%%%\raisebox{-3.8ex}{\includegraphics[height=1.2cm,keepaspectratio]{diag_2loop1.ps}}
%%%%\quad + \quad 
%%%%\raisebox{-3.8ex}{\includegraphics[height=1.2cm,keepaspectratio]{diag_2loop2.ps}}
%%%%~,
%%%%\]
%%%%%
%%%%\FromSlide{3}
%%%%	which by some standard superFeynman technique can be transformed
%%%%	into
%%%%%
%%%%\[
%%%%\raisebox{-4.0ex}{\includegraphics
%%%%	[height=1.2cm,keepaspectratio]{diag_2loop3.ps}}
%%%%\quad + \quad 
%%%%\raisebox{-4.0ex}{\includegraphics
%%%%	[height=1.2cm,keepaspectratio]{diag_2loop4.ps}}
%%%%\fromSlide{4}{
%%%%	\quad = \quad 0~.
%%%%	}
%%%%\]
%%%%
%%%%%%\FromSlide{4}
%%%%%%\[
%%%%%%\raisebox{-4.0ex}{\includegraphics
%%%%%%	[height=1.2cm,keepaspectratio]{diag_2loop3.ps}}
%%%%%%\quad + \quad 
%%%%%%\raisebox{-4.0ex}{\includegraphics
%%%%%%	[height=1.2cm,keepaspectratio]{diag_2loop4.ps}} 
%%%%%%	\quad = \quad 0~.
%%%%%%\]
%%%%%
%%%%\FromSlide{5}
%%%%	In general, if one uses the {\myit background superfield
%%%%	formalism}, any tadpole diagram will be an identical zero:
%%%%\[
%%%%	\raisebox{-3.5ex}{
%%%%	\includegraphics[height=1.0cm,keepaspectratio]{diag_FI2.ps}}
%%%%	\quad \equiv 0 ~.
%%%%\]
%%%%
%%%%\end{slide}
%%%%}
%%%%
%%%%%%%%%%%%%%%%%%%%%%%%%%%%%%%%  SLIDE %%%%%%%%%%%%%%%%%%%%%%%%%%%%%%%%%%%%%%%
%%%%
%%%%\overlays{3}
%%%%{
%%%%\begin{slide}[Replace]{ Absence of Gauge Anomaly }
%%%%	
%%%%\FromSlide{2}
%%%%	One can use the technique developed by Fujikawa and Konishi.
%%%%
%%%%	Our classical LV action is
%%%%%
%%%%\begin{equation*}
%%%%S ~=~ \int d^8z\, \overline{\Phi} 
%%%%e^V \Big(1 ~+~ i {\red N^\mu}\, \nabla_\mu  \Big) \Phi~, 
%%%%\end{equation*}
%%%%%
%%%%	
%%%%\FromSlide{3}
%%%%	One considers the variation of the effective action obtained
%%%%	by integrating out the chiral superfield
%%%%	{\blue $ \Phi_+ $}, under a chiral gauge transformation
%%%%	{\blue $ \delta \Lambda $}:
%%%%
%%%%%
%%%%\begin{equation*}
%%%%\delta_\Lambda\Gamma(V) ~=~ 
%%%%\langle \delta_\Lambda S \rangle = 
%%%%\Big\langle \int d^8z\,  \overline{\Phi} 
%%%%e^V \Big(1 ~+~ i N^\mu \nabla_\mu  \Big) (\delta \Lambda\, \Phi)
%%%%\Big\rangle~.
%%%%%\label{vareffact}
%%%%\end{equation*} 
%%%%%
%%%%
%%%%	Plugging in an effective propagator for {\blue $ \Phi_+$}
%%%%	in the presence of the background field {\blue $ V $}
%%%%	one then notes that the LV part completely cancels out
%%%%	leaving the gauge variation the same as in the Lorentz
%%%%	invariant theory.
%%%%
%%%%\end{slide}
%%%%}
%%%%
%%%%%%%%%%%%%%%%%%%%%%%%%%%%%%%%  SLIDE %%%%%%%%%%%%%%%%%%%%%%%%%%%%%%%%%%%%%%%
%%%%
%%%%\overlays{5}
%%%%{
%%%%\begin{slide}[Replace]{RG Evolution of LV Parameters}
%%%%\vspace{-0.5cm}
%%%%	For renormalization of the LV parameters of the dimension 5
%%%%	operators, one computes the following diagrams:
%%%%\FromSlide{2}
%%%%\begin{center}
%%%%\begin{tabular}{cccc}
%%%%\includegraphics[width=2.0cm,keepaspectratio]{diag_chiral_B.ps}
%%%%&
%%%%\includegraphics[width=2.0cm,keepaspectratio]{diag_chiral_D.ps}
%%%%&
%%%%\includegraphics[width=2.0cm,keepaspectratio]{diag_chiral_A.ps}
%%%%&
%%%%\includegraphics[width=2.0cm,keepaspectratio]{diag_chiral_E.ps}
%%%%\end{tabular}
%%%%\end{center}
%%%%	in the chiral sector and
%%%%\FromSlide{3}
%%%%\begin{center}
%%%%\begin{tabular}{cc}
%%%%\includegraphics[width=2.2cm,keepaspectratio]{diag_gauge_A.ps}
%%%%&
%%%%\includegraphics[width=2.2cm,keepaspectratio]{diag_gauge_F.ps}
%%%%\end{tabular}
%%%%\end{center}
%%%%	in the gauge sector.
%%%%
%%%%\FromSlide{4}
%%%%	One then obtains and solves the following RG equation:
%%%%%%
%%%%%% Undiagonalized RG equation
%%%%%%
%%%%\begin{equation*}
%%%%%\label{RG_eqn_undiag}
%%%%     \mu \frac{\partial}
%%%%              {\partial\mu} 
%%%%                \left(
%%%%\begin{array}{c}
%%%%               \red{N^\nu} \\ 
%%%%   \red{N_+^\nu} \\
%%%%                   \red{N_{-}^\nu} \\
%%%%   \red{T^{\mu\nu\rho}}
%%%%                \end{array} \right) ~=~  
%%%%     \frac{\alpha}
%%%%          {2 \pi} \, 
%%%%     \left(\begin{array}{rrrr}
%%%%                    2 & -1 & -1 & ~~0 \\
%%%%   -6 &  3 &  0 & ~~0 \\
%%%%                   -6 &  0 &  3 & ~~0 \\
%%%%    0 &  0 &  0 & ~~2
%%%%           \end{array}\right)
%%%%     \left(
%%%%  \begin{array}{c}
%%%%                 \red{N^\nu} \\ 
%%%% \red{N_+^\nu} \\
%%%%                 \red{N_{-}^\nu} \\
%%%% \red{T^{\mu\nu\rho}}
%%%%          \end{array} \right)~.
%%%%\end{equation*}
%%%%%
%%%%\FromSlide{5}
%%%%	For RG evolving the parameters from 
%%%%{\red $M = M_{\rm Pl} \approx 10^{19}$ GeV} to
%%%%{\red $\mu = m_{s} \approx 1$ TeV} one obtains a change in the 
%%%%LV parameters of only about {\red 10\%}.
%%%%
%%%%\end{slide}
%%%%}
%%%%
%%%%%%%%%%%%%%%%%%%%%%%%%%%%%%%%  SLIDE %%%%%%%%%%%%%%%%%%%%%%%%%%%%%%%%%%%%%%%
%%%%
%%%%\overlays{6}
%%%%{
%%%%\begin{slide}{ Soft Supersymmetry Breaking }
%%%%\onlySlide*{1}{
%%%%\DefaultTransition{Dissolve}
%%%%}
%%%%\fromSlide*{2}
%%%%{
%%%%\DefaultTransition{Replace}
%%%%}
%%%%
%%%%	We give a mass splitting to the scalar electron (and positron),
%%%%	and ignore gaugino masses.
%%%%\FromSlide{2}
%%%%	SUSY breaking masses for the electron and positron can be 
%%%%	written as
%%%%%%
%%%%%% SB vertex
%%%%\begin{equation*}
%%%%%\label{SB_vertex}
%%%%  \mathcal{L}_{\rm SB} ~=~  
%%%%- \int d^2\theta ~ \theta^2~ 
%%%%({\red m_{s}^0})^2 \, \Phi_+ \Phi_-
%%%%~+~ \text{h.c.} 
%%%%~- \int d^4\theta ~
%%%%\theta^2\overline{\theta}{}^2~ 
%%%%\Big\lgroup 
%%%%({\red m_s^+})^2\, \overline{\Phi}_+ \Phi_+ 
%%%%~+~
%%%%({\red m_s^-})^2\, \overline{\Phi}_- \Phi_-
%%%%\Big\rgroup  
%%%%~. 
%%%%\end{equation*}
%%%%%
%%%%\FromSlide{3}
%%%%	Once SUSY is broken, dimension 3 operators are allowed.
%%%%\FromSlide{4}
%%%%	Generically, in the SQED field content, written a la
%%%%	Kostelecky, there are the following dimension 3 interactions:
%%%%%
%%%%\begin{eqnarray*}
%%%%% first line
%%%%%\nonumber
%%%%{\cal L}_{\rm SB~LV~dim~3}^{\rm matter} 
%%%%&~=~& 
%%%%2\;i\, {\dgreen \wt{A}_+^\mu}\, \overline{z}_+ \mathcal{D}_\mu z_+ 
%%%%~+~ 2\;i\, {\dgreen \wt{A}_-^\mu}\, \overline{z}_-\mathcal{D}_\mu z_- 
%%%%~+~i\, {\dgreen \wt{C}^\mu}\, z_- \mathcal{D}_\mu z_+ 
%%%%%\label{LV_dim3_comp}
%%%%\\[1ex] 
%%%%% third line
%%%%\nonumber
%%%%&& 
%%%%~+~{\dgreen \wt{B}_+^\mu}\, \overline{\psi}_+\overline{\sigma}_\mu \psi_+ 
%%%%~+~ {\dgreen \wt{B}_-^\mu}\, \overline{\psi}_-\overline{\sigma}_\mu \psi_-
%%%% ~+~{\dgreen \wt{D}^{\mu\nu}}\, \psi_- \sigma_{\mu\nu} \psi_+~
%%%%\end{eqnarray*}
%%%%%
%%%%	in the matter sector and 
%%%%\FromSlide{5}
%%%%%%
%%%%%% Gauge dimension 3 operators in components
%%%%\begin{eqnarray*}
%%%%{\cal L}_{\rm SB~LV~dim~3}^{\rm gauge} ~=~ 
%%%%\onlySlide*{5}{
%%%%	\rnode{A}{
%%%%	{\dgreen \wt{E}_\mu}\, \epsilon^{\mu\nu\rho\sigma}\, 
%%%%	}
%%%%	A_\nu \partial_\rho A_\sigma  
%%%%}
%%%%\fromSlide*{6}{
%%%%	\rnode{A}{
%%%%	{\dgreen \wt{E}_\mu}\, {\black \epsilon^{\mu\nu\rho\sigma}}\, 
%%%%	}
%%%%	{\black A_\nu \partial_\rho A_\sigma}
%%%%}
%%%%~+~ 
%%%%{\dgreen \wt{F}_\mu}\, \lambda \sigma^\mu \overline{\lambda} 
%%%%~
%%%%\end{eqnarray*}
%%%%%
%%%%	in the gauge sector, 
%%%%\FromSlide{6}
%%%%	including the \rnode{B}{Chern-Simons} term.
%%%%\fromSlide*{6}{\ncline[linecolor=red,nodesep=0.5mm]{->}{B}{A}}
%%%%
%%%%\end{slide}
%%%%}
%%%%
%%%%%%%%%%%%%%%%%%%%%%%%%%%%%%%%  SLIDE %%%%%%%%%%%%%%%%%%%%%%%%%%%%%%%%%%%%%%%
%%%%
%%%%\overlays{4}
%%%%{
%%%%\begin{slide}[Replace]{ Dimension 3 Operators in the Matter Sector }
%%%%
%%%%	There are two types of reduction of operators dimension 3 
%%%%	$ \to $ dimension 5:
%%%%%% mechanisms of dim 5 -> dim 3 reduction
%%%%\begin{equation}
%%%%\begin{array}{l c r l} 
%%%%\, [LV]_{\rm dim~5} & ~\stackrel{\mathrm {EOM}}{\longrightarrow}~ &
%%%%  (m_{s}^2 + m_e^2)\, [LV]_{\rm dim~3}~,~ &{\rm for~selectrons}~, 
%%%%\\[1ex] \,
%%%%[LV]_{\rm dim~5} &~ \stackrel{\mathrm {1\ loop}}{\longrightarrow}~ &
%%%%  m_{s}^2\, [LV]_{\rm dim~3}~,~ & {\rm for~fermions ~and ~vector~bosons}
%%%%~.
%%%%\end{array}
%%%%\nonumber
%%%%\end{equation}
%%%%%
%%%%\FromSlide{2}
%%%%	Phenomenologically interesting are seemingly only the operators
%%%%%
%%%%\begin{eqnarray*}
%%%%% first line
%%%%%\nonumber
%%%%\lefteqn{
%%%%	{\cal L}_{\rm dim~3~phen}^{\rm matter} =
%%%%	}
%%%%	\\
%%%%% second line
%%%%	&& =~
%%%%  {\dgreen \wt{B}_+^\mu}\, \overline{\psi}_+\overline{\sigma}_\mu \psi_+ 
%%%%~+~ {\dgreen \wt{B}_-^\mu}\, \overline{\psi}_-\overline{\sigma}_\mu \psi_-
%%%%\fromSlide{3}{
%%%% ~+~ 2\;i\, {\dgreen \wt{A}_+^\mu}\, \overline{z}_+ \mathcal{D}_\mu z_+ 
%%%%~+~ 2\;i\, {\dgreen \wt{A}_-^\mu}\, \overline{z}_-\mathcal{D}_\mu z_- 
%%%%	~.}
%%%%\end{eqnarray*}
%%%%%
%%%%\FromSlide{4}
%%%%	On the equations of motion, dimension 5 operators induce
%%%% \begin{equation*}
%%%%{\dgreen \wt{A}_\pm^\mu} ~=~  
%%%%\pm\, 2\, \frac{{\red N_V^\mu}}{ M }  \, 
%%%%(m_e^2 \, +\,  m_s^2)~.
%%%%\end{equation*}
%%%%
%%%%\end{slide}
%%%%}
%%%%
%%%%%%%%%%%%%%%%%%%%%%%%%%%%%%%%  SLIDE %%%%%%%%%%%%%%%%%%%%%%%%%%%%%%%%%%%%%%%
%%%%
%%%%\overlays{3}
%%%%{
%%%%\begin{slide}[Replace]{ 1-loop Corrections to Matter Dimension 3 Operators }
%%%%
%%%%	Including 1-loop corrections induced by dimension 5 operators
%%%%\vspace{0.3cm}
%%%%\begin{center}
%%%%\includegraphics[width=2.4cm,height=2.7cm,keepaspectratio]
%%%% {diag_chiral_SB_chiral_LV_A.ps} 
%%%%%\includegraphics[width=2.7cm,height=2.7cm,keepaspectratio]
%%%%% {diag_chiral_SB_chiral_LV_B.ps} 
%%%%\includegraphics[width=2.4cm,height=2.7cm,keepaspectratio]
%%%% {diag_chiral_SB_chiral_LV_C.ps} 
%%%%%\includegraphics[width=2.7cm,height=2.7cm,keepaspectratio]
%%%%% {diag_chiral_SB_chiral_LV_D.ps}
%%%%\includegraphics[width=2.4cm,height=2.7cm,keepaspectratio]
%%%% {diag_chiral_SB_gauge_LV.ps}
%%%%\end{center}
%%%%\FromSlide{2}
%%%%	one arrives at the RG equations
%%%%%%                  
%%%%%% RG equations for A^\mu and B^\mu
%%%%\begin{eqnarray*}
%%%%%\label{RG_AB}
%%%%% A^\mu
%%%%\nonumber
%%%%\mu\frac{d {\dgreen \wt{A}_+^\nu}}{d\mu} 
%%%%& ~=~ &
%%%%\frac{\alpha}{\pi} \,  \Big\lgroup  {\dgreen \wt{A}_+^\nu}
%%%%~-~{\dgreen \wt{B}_+^\nu} \Big\rgroup~, 
%%%%        \\
%%%%% B^\mu
%%%%\mu\frac{d {\dgreen \wt{B}_+^\nu}}{d\mu} 
%%%%& ~=~ &
%%%%\frac{\alpha}{2\pi} \, \Big\lgroup
%%%%{\dgreen \wt{B}_+^\nu}  ~-~ {\dgreen \wt{A}_+^\nu}  
%%%%~+~3\, \frac{(m_s^+)^2}{M}\, {\red N^\nu}
%%%%~-~2\, \frac{(m_s^+)^2}{M}\, {\red N_+^\nu}
%%%%\Big\rgroup~,
%%%%\end{eqnarray*}
%%%%%
%%%%\FromSlide{3}
%%%%	with a leading $ \alpha \log $ solution
%%%%%%
%%%%%% The induced coefficient B_\mu
%%%%\begin{equation*}
%%%%{\dgreen \wt{B}^{\pm\nu}} (m_s) ~=~ \frac{\alpha}{\pi}\, 
%%%%\log ( M/m_s )\,
%%%%\frac{(m_{s}^\pm)^2 }{M}\, 
%%%%\Bigg\{ 
%%%%\frac{3}
%%%%     {2} {\red N^\nu}(M) 
%%%%~-~ {\red N_\pm^{\,\nu}}(M)
%%%%\Bigg\}~.
%%%%\end{equation*}
%%%%%
%%%%
%%%%\end{slide}
%%%%}
%%%%
%%%%%%%%%%%%%%%%%%%%%%%%%%%%%%%%  SLIDE %%%%%%%%%%%%%%%%%%%%%%%%%%%%%%%%%%%%%%%
%%%%
%%%%\overlays{4}
%%%%{
%%%%\begin{slide}[Replace]{ 1-loop Induced Gauge Dimension 3 Operators }
%%%%
%%%%	The answer for gauge operators is given by
%%%%\vspace{0.5cm}
%%%%\begin{center}
%%%%\includegraphics[width=2.3cm,keepaspectratio]
%%%% {diag_gauge_SB_chiral_LV_A.ps} 
%%%%\includegraphics[width=2.3cm,keepaspectratio]
%%%% {diag_gauge_SB_chiral_LV_C.ps} 
%%%%\includegraphics[width=2.3cm,keepaspectratio]
%%%% {diag_gauge_SB_chiral_LV_D.ps} 
%%%%\includegraphics[width=2.3cm,keepaspectratio]
%%%% {diag_gauge_SB_chiral_LV_F.ps} 
%%%%\includegraphics[width=2.3cm,keepaspectratio]
%%%% {diag_gauge_SB_chiral_LV_H.ps} 
%%%%\end{center}
%%%%\vspace{0.5cm}
%%%%\FromSlide{2}
%%%%	If one is interested only in the observable sector, 
%%%%	{\myit i.e.} photons, the result is
%%%%\FromSlide{3}
%%%%\[
%%%%	{\dgreen \wt{E}^\mu} ~=~ 0~.
%%%%\]
%%%%\FromSlide{4}
%%%%	The {\myit Chern-Simons term is not generated} whether in
%%%%	a SUSY or in a SUSY-breaking theory.
%%%%
%%%%	This is also confirmed by the Coleman-Glashow theorem.
%%%%
%%%%\end{slide}
%%%%}
%%%%
%%%%%%%%%%%%%%%%%%%%%%%%%%%%%%%%  SLIDE %%%%%%%%%%%%%%%%%%%%%%%%%%%%%%%%%%%%%%%
%%%%
%%%%\overlays{5}
%%%%{
%%%%\begin{slide}[Replace]{ Phenomenology: Component Form of the Chiral Operators }
%%%%
%%%%\vspace{-0.5cm}
%%%%	In order to put limits on the dimension 5 LV operators, one
%%%%	obtains a component form of them.
%%%%\FromSlide{2}
%%%%	The chiral operator 
%%%%$ {\red {N_+^\mu}}\, \overline{\Phi}_+ e^{2eV} i \nabla_\mu \Phi_+ $
%%%%	(and similarly the one with {\red $ N_-^\mu $}) 
%%%%	can be expanded into
%%%%\FromSlide{3}
%%%%%%
%%%%%% The electron LV operator in components with Weyl spinors; 
%%%%%% totally unresolved
%%%%\begin{gather*}
%%%%% the 1st line
%%%%\nonumber
%%%%  \mathcal{L}_{\mathrm{LV~dim~5}}^{\mathrm{matter\,(+)}} 
%%%%~ =~ \frac{\red N_+^\mu}{M} \Big[~
%%%%    i \bar{F}_+ \mathcal{D}_\mu F_+ ~+~
%%%%    i e \bar{z}_+ D \mathcal{D}_\mu z_+ ~-~
%%%%    i e \mathcal{D}_\mu(\bar{z}_+) D z_+ 
%%%%~+~ 
%%%%  \frac{1}{2}\bar{\psi}_+\mathcal{D}_{(\mu}\mathcal{D}_{\nu)}
%%%%               \bar{\sigma}^\nu \psi_+ 
%%%%\nonumber \\
%%%%% the 3rd line
%%%%  + ~ 
%%%%    i e \frac{\sqrt{2}}{2} \Big\{
%%%%               \overline{\psi}_+\bar\sigma_\mu\lambda F_+ 
%%%%       ~-~
%%%%               \overline{F}_+\overline{\lambda} \bar\sigma_\mu \psi_+
%%%%                         \Big\}  ~+~
%%%%    e^2 \bar{z}_+ \Big\{
%%%%               \lambda\sigma_\mu\bar{\lambda} 
%%%%       ~-~
%%%%               \overline{\lambda}\bar\sigma_\mu\lambda 
%%%%                       \Big\} z_+ 
%%%%~+~ 
%%%%    \frac{1}{2} e \overline{\psi}_+\bar\sigma_\mu D\psi_+
%%%%% the 5th line
%%%%\nonumber \\
%%%%\nonumber
%%%% -~ 
%%%%   \sqrt{2} e \Big\{ 
%%%%                     \mathcal{D}_\mu(\overline{\psi}_+)\overline{\lambda} z_+ 
%%%%     ~+~ 
%%%%                     \bar{z}_+ \lambda \mathcal{D}_\mu \psi_+ 
%%%%                     \Big\} 
%%%%    ~-~ 
%%%%    \frac{\sqrt{2}}{2} e \Big\{ 
%%%%                      \overline{\psi}_+\bar\sigma^\nu\sigma_\mu 
%%%%                     \bar{\lambda}\mathcal{D}_\nu z_+ +
%%%%                     \mathcal{D}_\nu(\bar{z}_+)\lambda\sigma_\mu
%%%%                     \bar{\sigma}^\nu \psi_+
%%%%                     \Big\}
%%%%   \\
%%%%% the 7th line
%%%%  ~ -~
%%%%  \frac{1}{4} e \bar{\psi}_+\epsilon_\mu{}^{\nu\rho\sigma}
%%%%              F_{\rho\sigma} \bar{\sigma}_\nu \psi_+
%%%%   ~+~
%%%%  i \bar{z_+} \mathcal{D}^\nu \mathcal{D}_\mu \mathcal{D}_\nu z_+ 
%%%%   ~+~
%%%%   \frac{1}{2} i e \mathcal{D}_\nu (\bar{z}_+) \epsilon_\mu{}^{\nu\rho\sigma}
%%%%              F_{\rho\sigma} z_+ \, 
%%%%   \Big] ~.
%%%%%\label{LV_electron_comp}
%%%%\end{gather*}
%%%%%
%%%%\FromSlide{4}
%%%%	On the equations of motion, in the observable sector this 
%%%%	produces
%%%%%%
%%%%%% Resolved LV operators in the quark sector in Dirac spinors
%%%%\begin{equation*}
%%%%%% first line
%%%%   \mathcal{L}_{\rm LV}^{\rm matter} ~=~ 
%%%%% 1st operator
%%%%        -
%%%%       \frac{{\red N_A^\mu}}
%%%%              {M} \, \frac{1}{2}e \,
%%%%       \overline{\Psi} \widetilde{F}_{\mu\nu}
%%%%% \epsilon_{\mu\nu\rho\sigma} F^{\rho\sigma}
%%%%                       \gamma_\nu \Psi 
%%%%% 2nd operator
%%%%     ~-~
%%%%        \frac{{\red N_V^\mu}}
%%%%              {M} \, \frac{1}{2}e \,
%%%%       \overline{\Psi} \widetilde{F}_{\mu\nu}
%%%%%\epsilon_{\mu\nu\rho\sigma} F^{\rho\sigma}
%%%%                       \gamma^\nu \gamma^5 \Psi 
%%%%% 3rd operator
%%%%~+~  \frac{{\red N_V^\mu}}{M} \,m_e^2\, \overline{\Psi} \gamma_\mu \Psi
%%%%     ~,
%%%%%\label{resolved_LV_Dirac}
%%%%\end{equation*}
%%%%%
%%%%\FromSlide{5}
%%%%	where the definite parity operators are
%%%%%%
%%%%%% definition of N_V, N_A
%%%%\begin{equation*}
%%%%      {\red N^\mu_V} ~=~  \frac{ {\red N_+^\mu} ~-~ {\red N_-^\mu} }{2}~,   
%%%%\qquad 
%%%%      {\red N^\mu_A} ~=~  \frac{ {\red N_+^\mu} ~+~ {\red N_-^\mu} }{2}~.
%%%%\end{equation*}
%%%%%
%%%%
%%%%\end{slide}
%%%%}
%%%%
%%%%%%%%%%%%%%%%%%%%%%%%%%%%%%%%  SLIDE %%%%%%%%%%%%%%%%%%%%%%%%%%%%%%%%%%%%%%%
%%%%
%%%%\overlays{6}
%%%%{
%%%%\begin{slide}[Replace]{ Phenomenology: Component Form of the Gauge Operators }
%%%%
%%%%	The $ D $-term operator 
%%%%$ \frac 1M \int d^4\theta \, 
%%%%{\red N^\kappa}\, \overline{W} \bar{\sigma}_\kappa W $
%%%%	written in components looks as 
%%%%\FromSlide{2}
%%%%%%
%%%%%% gauge Kahler term in components
%%%%\begin{equation*}
%%%%%\lefteqn{
%%%%\mathcal{L}_{\mathrm{LV~dim~5}}^{\mathrm{gauge\ (K)}} 
%%%% ~=~ \frac{\red N^\mu}{M}\,  
%%%%\Big\lgroup 
%%%%2\, \overline{\lambda}\,\gamma_\mu\, \Box\, 
%%%%   \lambda 
%%%%~+~
%%%%2\, \lambda\, \partial_\mu \slashed{\partial}\, 
%%%%   \overline{\lambda} 
%%%%~-~ 
%%%%2\, D\, \partial^\nu F_{\mu\nu}
%%%%~+~ 
%%%%\partial_\lambda F^{\lambda\nu}\, 
%%%%\widetilde{F}_{\nu\mu} 
%%%%\Big\rgroup~,
%%%%\end{equation*}
%%%%%
%%%%\FromSlide{3}
%%%%	which on the equations of motion turns into
%%%%%%                                             __
%%%%%% Observable operator induced on EOM from the WnW
%%%%\begin{equation*}
%%%%%\label{LV_induced_by_gauge_K}
%%%%        \mathcal{L}_{\rm gauge\ (K)}^{\rm EOM} ~=~  
%%%% e\, \overline{\Psi}\, {\red N^\mu} \gamma^\nu
%%%%\widetilde{F}_{\mu\nu}\, \Psi~.
%%%%\end{equation*}
%%%%%
%%%%\FromSlide{4}
%%%%	The tensor operator 
%%%%$ \frac 1{4M} 
%%%%\int d^2\theta \, {\red T^{\lambda\, \mu\nu}} \,
%%%%        W \sigma_{\mu\nu} \, \partial_\lambda W  
%%%%~+~ {\rm h.c.} ~$
%%%%	in components is
%%%%\FromSlide{5}
%%%%%%
%%%%%% gauge Tensor term in components
%%%%\begin{equation*}
%%%%% first line
%%%%\mathcal{L}_{\mathrm{LV~dim~5}}^{\mathrm{gauge\ (T)}}  
%%%%      ~  =~ 
%%%%\frac{2\,{\red T_{\mu\nu\rho}}}{M}
%%%%\Big( 
%%%%F^{\nu\lambda}\partial^\mu F^\rho_{\phantom{\rho}\lambda}
%%%%\,-\, D\, \partial^\mu \widetilde{F}^{\nu\rho} 
%%%%\,-\,\overline{\lambda}\, \partial^\mu \partial_\sigma
%%%%\overline{\sigma}_\tau \lambda\; \epsilon^{\sigma\tau\nu\rho}
%%%%\Big)~.
%%%%\end{equation*}
%%%%\FromSlide{6}
%%%%	At tree level this induces
%%%%%%
%%%%%% observable terms generated by reduction of the T-term
%%%%%% on the EOM
%%%%\begin{eqnarray*}
%%%%% first line
%%%%        \mathcal{L}_{\rm gauge\ (T)}^{\rm EOM} ~=~   
%%%%        2\,e\,  {\red T_{\mu\nu\rho}} \, 
%%%%         \overline{\Psi} \gamma^\mu F^{\nu\rho} \Psi
%%%%        ~
%%%%\end{eqnarray*}
%%%%%
%%%%	in the observable sector.
%%%%
%%%%\end{slide}
%%%%}
%%%%
%%%%%%%%%%%%%%%%%%%%%%%%%%%%%%%%  SLIDE %%%%%%%%%%%%%%%%%%%%%%%%%%%%%%%%%%%%%%%
%%%%
%%%%\overlays{3}
%%%%{
%%%%\begin{slide}{ Effective Lorentz Violating Lagrangian }
%%%%\onlySlide*{1}{
%%%%\DefaultTransition{Blinds}
%%%%}
%%%%\fromSlide*{2}
%%%%{
%%%%\DefaultTransition{Replace}
%%%%}
%%%%
%%%%	We can now bring all terms together and write out the
%%%%	effective lagrangian
%%%%%%
%%%%%% All operators of phenomenological interest
%%%%\begin{equation*}
%%%%%% first line
%%%%%\label{L_eff}
%%%% - \mathcal{L}_{\rm eff~LV}
%%%%         ~=~ 
%%%%\overline{\Psi}\, \gamma^\mu \Big\lgroup 
%%%% {\dgreen a_\mu} ~+~ {\dgreen b_\mu} \gamma^5 
%%%%~+~ e\,{\dgreen c_\nu}\,   \wt{F}^{\nu\mu} 
%%%%~+~  e\, {\dgreen d_\nu}\,\wt{F}^{\nu\mu} \gamma^5
%%%%~+~e\, {\dgreen f_{\mu\rho\sigma}}\,  F^{\rho\sigma} 
%%%%\Big\rgroup \Psi~, 
%%%%\end{equation*} 
%%%%%
%%%%	where the LV parameters, normalized at $ m_{s} $ are
%%%%\FromSlide{2}
%%%%%%
%%%%%% Coefficients of the effective lagrangian
%%%%\begin{eqnarray*}
%%%%%% first line
%%%%\nonumber
%%%%        {\dgreen a^\mu} & ~=~ &
%%%%         -\, \frac{1}{M}\, m_e^2 {\red N_V^\mu}
%%%%        ~+~
%%%%        \frac{\alpha\, \log (M/m_s)}{\pi M}\, 
%%%%        \Big\{
%%%%%        \frac{e^2}{4\pi^2}\, m_s^2\, N_-^\mu 
%%%%         m_s^2\, {\red N_V^\mu }
%%%%                ~+~
%%%%%\frac{e^2}{4\pi^2}\, \frac{\Delta m^2}{2}\, 
%%%%                 \frac{\Delta m^2}{2}\, 
%%%%                                   {\red     N_A^\mu}
%%%%                ~-~
%%%%%% second line
%%%%%\frac{3e^2}{8\pi^2}\, \frac{\Delta m^2}{2}\, 
%%%%                \frac{3}{2}\, \frac{\Delta m^2}{2}\, 
%%%%                                   {\red       N^\mu }
%%%%        \Big\}~ ,
%%%%\\
%%%%%\label{L_eff_coefs}
%%%%%% third line
%%%%        {\dgreen b^\mu} & ~=~ & 
%%%%        \frac{\alpha\,\log (M/m_s)}{\pi M} \, 
%%%%        \Big\{
%%%%%        \frac{e^2}{4\pi^2}\, m_s^2\, N_+^\mu
%%%%                 m_s^2\, {\red N_A^\mu}
%%%%                ~+~
%%%%%\frac{e^2}{4\pi^2}\, \frac{\Delta m^2}{2}\, 
%%%%                 \frac{\Delta m^2}{2}\, 
%%%%                                       {\red N_V^\mu}
%%%%                ~-~
%%%%%\frac{3e^2}{8\pi^2}\, m_s^2\, n^\mu
%%%%                \frac{3}{2}\, m_s^2\, {\red N^\mu}
%%%%        \Big\} ~,
%%%%\\
%%%%%% fourth line
%%%%\nonumber
%%%%        {\dgreen c^\mu} & ~=~ &
%%%%         \frac{1}{M}
%%%%        \Big\{ 
%%%%                \frac{1}{2}{\red N_A^\mu}
%%%%                ~-~
%%%%                {\red N^\mu}
%%%%        \Big\}~,\qquad 
%%%%        {\dgreen d^\mu} ~=~
%%%%        \frac{1}{M}\, \frac{\red N_V^\mu}{2} 
%%%%~,\qquad 
%%%%        {\dgreen f^{\mu\nu\rho}} ~ = ~
%%%%        \frac{2}{M} {\red T^{\mu\nu\rho}}
%%%%~.
%%%%\end{eqnarray*}
%%%%%
%%%%\FromSlide{3}
%%%%	In QED, the operator $ {\dgreen a^\mu} $ can be totally
%%%%	excluded via a phase redefinition
%%%%$ \Psi(x) \to e^{i {\dgreen a^\mu} x_\mu} \Psi(x) $.
%%%%\end{slide}
%%%%}
%%%%
%%%%%%%%%%%%%%%%%%%%%%%%%%%%%%%%  SLIDE %%%%%%%%%%%%%%%%%%%%%%%%%%%%%%%%%%%%%%%
%%%%
%%%%\overlays{7}
%%%%{
%%%%\begin{slide}{ Effective Hamiltonian }
%%%%\onlySlide*{2}{
%%%%\DefaultTransition{Box}
%%%%}
%%%%\fromSlide*{3}{
%%%%\DefaultTransition{Replace}
%%%%}
%%%%\onlyInPS{
%%%%\vspace{-0.7cm}
%%%%}
%%%%	In order to infer phenomenological consequences one needs to
%%%%	derive a non-relativistic effective Hamiltonian corresponding
%%%%	to the lagrangian
%%%%\onlySlide*{1}
%%%%{
%%%%    %%
%%%%    %% All operators of phenomenological interest
%%%%    \begin{equation*}
%%%%    %% first line
%%%%    %\label{L_eff}
%%%%     - \mathcal{L}_{\rm eff~LV}
%%%%             ~=~ 
%%%%    \overline{\Psi}\, \gamma^\mu \Big\lgroup 
%%%%     {\dgreen a_\mu} ~+~ {\dgreen b_\mu} \gamma^5 
%%%%    ~+~ e\,{\dgreen c_\nu}\,   \wt{F}^{\nu\mu} 
%%%%    ~+~  e\, {\dgreen d_\nu}\,\wt{F}^{\nu\mu} \gamma^5
%%%%    ~+~e\, {\dgreen f_{\mu\rho\sigma}}\,  F^{\rho\sigma} 
%%%%    \Big\rgroup \Psi~ 
%%%%    \end{equation*} 
%%%%    %
%%%%}
%%%%\FromSlide{2}
%%%%\PDFtransition{Glitter}
%%%%%%
%%%%%% Low energy nonrelativistic Hamiltonian for the 
%%%%%% low energy effective lagrangian
%%%%\begin{eqnarray*}
%%%%%% first line
%%%%        \mathcal{H}_{\rm eff} 
%%%%        & ~=~ &
%%%%% a-operator
%%%%        \frac{\vec{p} \cdot {\dgreen \vec{a}}}
%%%%                  {m}
%%%%        ~+~
%%%%% b-operator
%%%%        {\dgreen \vec{b}} \cdot \vec{\sigma}
%%%%        ~+~
%%%%% c-operator
%%%%        \Big\{\, 
%%%%                \frac{e\, \vec{p}}
%%%%                    {m}
%%%%                \, ,\, 
%%%%                \big[ {\dgreen \vec{c}} \times \vec{E} \big]
%%%%                ~-~
%%%%                {\dgreen c^0}\, \vec{B} 
%%%%        \Big\}
%%%%        ~-~ 
%%%%        e \, {\dgreen d^0} \, \big( \vec{B} \cdot \vec{\sigma} \big)
%%%%\nonumber        \\
%%%%%% second line
%%%%%\label{H_eff}
%%%%% d-operator
%%%%        & &
%%%%        ~+~
%%%%        e\, {\dgreen \vec{d}} \cdot
%%%%        \big[ \vec{E} \times \vec{\sigma} \big]
%%%%        ~+~
%%%%% f-operator
%%%%        \Big\{\, 
%%%%                \frac{e\, p^k}
%%%%                    {m}
%%%%                \,,\, 
%%%%                2\, {\dgreen f_{k0l}}\, E^l 
%%%%                ~+~
%%%%                {\dgreen f_{klm}} \,\epsilon_{lmn}\, 
%%%%                B^n
%%%%        \Big\}~. 
%%%%\end{eqnarray*}
%%%%%
%%%%\FromSlide{3}
%%%%	The tightest contraints come from the experiments searching
%%%%	for abnormal spin precession around external directions 
%%%%	determined by the LV vectors.
%%%%	The parameter to compare with is the energy shift due
%%%%	to Lorentz violation --- $\Delta \omega_{\rm LV}$.
%%%%\FromSlide{4}
%%%%	Since the effects are mediated by dimension {\myit five}
%%%%	operators, the strength of the constraints greatly depends
%%%%	on the ``intrinsic'' energy scale $ \mu $ of the experiment:
%%%%$\Delta \omega_{\rm LV} \sim \mu^2 M^{-1}$.
%%%%	
%%%%	There are a few energy scales pertinent to 
%%%%$ {\mathcal H}_{\rm eff} $:
%%%%\begin{itemstep}
%%%%\FromSlide{5}
%%%%\item	the soft breaking scale $ m_{s} $,
%%%%\FromSlide{6}
%%%%\item	the hadronic scale $ \Lambda_{\rm QCD} $
%%%%\FromSlide{7}
%%%%\item   and the energy scale of the external electromagnetic field.
%%%%\end{itemstep}
%%%%
%%%%\end{slide}
%%%%}
%%%%
%%%%%%%%%%%%%%%%%%%%%%%%%%%%%%%%  SLIDE %%%%%%%%%%%%%%%%%%%%%%%%%%%%%%%%%%%%%%%
%%%%
%%%%\overlays{7}
%%%%{
%%%%\begin{slide}{ Main Constraints }
%%%%\onlySlide*{1}{
%%%%\DefaultTransition{Glitter}
%%%%}
%%%%\fromSlide*{2}
%%%%{
%%%%\DefaultTransition{Replace}
%%%%}
%%%%\begin{tabular}{p{5.4cm}|c}
%%%%\hline
%%%%	Type of Experiment & Estimate \\
%%%%\hline\hline
%%%%\fromSlide{2}{
%%%%%\DefaultTransition{Replace}
%%%%	Electron Spin Precession and Torsion Balance  } & 
%%%%\fromSlide{2}{ $ |{\red N_A^i} - \frac{3}{2} {\red N^i} | < 10^{-12} $  }\\ 
%%%%	& \\
%%%%\fromSlide{3}{
%%%%	Nuclear Spin Precession } &
%%%%\fromSlide{3}{
%%%%$ \frac {10^{19}~{\rm GeV}}{M} ~ |{\red N^i}| ~<~ 10^{-9} $ }\\
%%%%	& \\
%%%%\fromSlide{4}{
%%%%	LV Precession of the angular momentum of a 
%%%%	paramagnetic atom } &
%%%%\fromSlide{4}{
%%%%$ \frac {10^{19}~{\rm GeV}}{M} ~ |{\red N^i}-
%%%%				  {\red N_A^i}/2|~<~10^{-2} $ }\\
%%%%	& \\
%%%%\fromSlide{5}{
%%%%	CPT-odd anomalous magnetic moment of electron and
%%%%	positron } &
%%%%\fromSlide{5}{
%%%%$ \frac {10^{19}~{\rm GeV}}{M} ~ |{\red N_V^0}| ~<~ 10^{10} $} \\
%%%%\hline 
%%%%\end{tabular}
%%%%
%%%%\vspace{0.35cm}
%%%%\FromSlide{6}
%%%%	Despite the Ferrara-Rimiddi theorem, there {\myit is}
%%%%	anomalous magnetic moment of electron induced.
%%%%	The reason is that one of the theorem's assumptions ---
%%%%	Lorentz invariance --- is broken here.
%%%%
%%%%\FromSlide{7}
%%%%	Dimension six operators were not subject of detailed examination.
%%%%	Rough estimates suggest only $ M \sim 10^{14}~{\rm GeV} $,
%%%%	which is lower than the Planck scale.
%%%%	More detailed study is desired.
%%%%
%%%%\end{slide}
%%%%}
%%%%
%%%%%%%%%%%%%%%%%%%%%%%%%%%%%%%%  SLIDE %%%%%%%%%%%%%%%%%%%%%%%%%%%%%%%%%%%%%%%
%%%%
%%%%\overlays{15}
%%%%{
%%%%\begin{slide}{ Conclusions }
%%%%\onlySlide*{1}{
%%%%\DefaultTransition{Box}
%%%%}
%%%%\fromSlide*{2}
%%%%{
%%%%\DefaultTransition{Replace}
%%%%}
%%%%
%%%%\onlyInPS{
%%%%\vspace{-1.0cm}
%%%%}
%%%%
%%%%\begin{itemstep}
%%%%\untilSlide*{10}{
%%%%\item We have constructed a dimension five Lorentz-violating extension
%%%%	of SQED
%%%%%, as a subset of MSSM
%%%%}
%%%%\untilSlide*{11}{
%%%%\item obtained the list of CPT-even dimension six LV interactions
%%%%}
%%%%\untilSlide*{12}{
%%%%\item shown that no $ D $-term or gauge anomaly is induced due
%%%%	to introduction of LV terms 
%%%%%in the limit of the full SUSY
%%%%}
%%%%\untilSlide*{13}{
%%%%\item derived 1-loop RG equations for the dimension five LV interactions
%%%%}
%%%%\untilSlide*{14}{
%%%%\item calculated the dimension 3 operators induced by soft SUSY breaking
%%%%}
%%%%\item argued that quadratic divergencies are stabilized at the
%%%%	SUSY breaking scale, which alleviates the naturalness problem
%%%%\item proved that the Chern-Simons term is not generated at the loop
%%%%	level
%%%%\item demonstrated that none of the LV operators lead to high-energy
%%%%	modification of the dispersion relations
%%%%\item computed explicit component expression for the LV interactions
%%%%\item limited a combination of the vector backgrounds by using
%%%%	results of torsion balance experiments, at the level 
%%%%	$ O(10^{-12}) $
%%%%\end{itemstep}	
%%%%\dgreen
%%%%\FromSlide{11}
%%%%	Severe constraints pose a serious challenge for theories
%%%%	predicting Lorentz violation at $ 1/M_{\rm Pl} $ level
%%%%
%%%%\FromSlide{12}
%%%%	This may be motivated by symmetry reasons, {\myit e.g.} CPT
%%%%
%%%%\FromSlide{13}
%%%%	At the next order,  $ 1/M_{\rm Pl}^2 $, CPT-conserving operators
%%%%	are not excluded
%%%%
%%%%\onlyInPDF{
%%%%\FromSlide{14}
%%%%	They do not lead to modifications of dispersion relations, hence
%%%%	strong astrophysical bounds are not applicable to them
%%%%}
%%%%
%%%%\FromSlide{15}
%%%%	Estimates come close to experimental sensitivity, and therefore
%%%%	deserve further study in the framework of LV MSSM
%%%%
%%%%\end{slide}
%%%%}
%%%%
%%%%%%%%%%%%%%%%%%%%%%%%%%%%%%%%  SLIDE %%%%%%%%%%%%%%%%%%%%%%%%%%%%%%%%%%%%%%%
%%%%
%%%%\overlays{1}
%%%%{
%%%%\begin{slide}[Blinds]{ Collaboration }
%%%%\onlyInPDF{
%%%%\vspace{-0.9cm}
%%%%\clearpage
%%%%%%\begin{figure}
%%%%%%\centering
%%%%%%\includegraphics[width=10.8cm]{IMGP2770.PS}
%%%%%%%\caption{Caption goes here}
%%%%%%\end{figure}
%%%%%\clearpage
%%%%
%%%%\vspace{-7cm}
%%%%}
%%%%{\red\fontsize{10pt}{18pt}\selectfont
%%%%\qquad P.A.Bolokhov, S.G.Nibbelink, M.Pospelov, hep-ph/0505029
%%%%
%%%%\qquad We thank L.Smolin, N.Arkani-Hamed, M.Voloshin and M.Shifman
%%%%
%%%%\qquad --- University of Guelph
%%%%
%%%%\qquad --- University of Victoria
%%%%
%%%%\qquad --- Perimeter Institute
%%%%
%%%%}
%%%%
%%%%\end{slide}
%%%%}
%%%%
%%%%%%%%%%%%%%%%%%%%%%%%%%%%%%%%%%  SLIDE %%%%%%%%%%%%%%%%%%%%%%%%%%%%%%%%%%%%%%%
%%%%%%
%%%%%%\overlays{10}
%%%%%%{
%%%%%%\begin{slide}[Blinds]{ Title }
%%%%%%
%%%%%%
%%%%%%\end{slide}
%%%%%%}
%%%%%%
%%%%%%%%%%%%%%%%%%%%%%%%%%%%%%%%%%  SLIDE %%%%%%%%%%%%%%%%%%%%%%%%%%%%%%%%%%%%%%%
%%%%

\end{document}
